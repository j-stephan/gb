\chapter{Einleitung}

\section{Aufgabenstellung}

Der FDK-Algorithmus ist ein weit verbreiteter Ansatz zur Rekonstruktion von kegelförmiger Computer-Tomographie. In
diesem Beleg soll untersucht werden:

\begin{itemize}
    \item Zusammenfassung des Forschungsstandes hinsichtlich der Parallelisierung / der Verwendung von CUDA
    \item Gegenüberstellung verschiedener Optimierungsziele (Time-to-solution, Occupancy)
    \item Variantenvergleich verschiedener Implementierungsstrategien
    \item Implementierung und Analyse einer dieser Strategien
\end{itemize}

\begin{itemize}
    \item Computertomographie ist gesellschaftsrelevant etc.
    \item FDK-Algorithmus ist wichtiger Bestandteil von CT
    \item Ausgangssituation am HZDR: uraltes FDK-Programm braucht mehrere Tage für eine Rekonstruktion (schlecht)
    \item Gesamtziel: Dieses Programm soll durch ein schnelleres und möglichst gut optimiertes abgelöst werden
    \item Aufgabenvorstellung Großer Beleg
\end{itemize}
