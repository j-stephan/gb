\chapter{Einleitung}

\section{Die gesellschaftliche Relevanz der Computertomographie}

\begin{itemize}
    \item CT in ihrer Bedeutung nur mit Röntgen vergleichbar
    \item Nobelpreise für die CT-Erfinder unterstreichen das
\end{itemize}

\section{Der Einsatz der Computertomographie am Helmholtz-Zentrum Dresden-Rossendorf}

\begin{itemize}
    \item Computertomographie ist gesellschaftsrelevant etc.
    \item FDK-Algorithmus ist wichtiger Bestandteil von CT
    \item Ausgangssituation am HZDR: uraltes FDK-Programm braucht mehrere Tage für eine Rekonstruktion (schlecht)
    \item Gesamtziel: Dieses Programm soll durch ein schnelleres und möglichst gut optimiertes abgelöst werden
\end{itemize}

\section{Aufgabenstellung}

Der \gls{fdk} ist ein weit verbreiteter Ansatz zur Rekonstruktion von kegelförmiger Computer-Tomographie. In diesem
Beleg soll untersucht werden:

\begin{itemize}
    \item Zusammenfassung des Forschungsstandes hinsichtlich der Parallelisierung / der Verwendung von \gls{cuda}
    \item Gegenüberstellung verschiedener Optimierungsziele (Time-to-solution, Occupancy)
    \item Variantenvergleich verschiedener Implementierungsstrategien
    \item Implementierung und Analyse einer dieser Strategien
\end{itemize}
