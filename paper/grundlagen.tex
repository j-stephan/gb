\chapter{Grundlagen}

\section{Der Feldkamp-Davis-Kress-Algorithmus}

\begin{itemize}
    \item Welches Problem wird gelöst und wo tritt dieses Problem auf? -> evtl. kurzer geschichtlicher Überblick zur CT
    \item Wie funktioniert der FDK-Algorithmus? -> Detaillierungsgrad?
    \item Wie sieht der Datenfluss aus? -> Detektor, Vorverarbeitung, Wichtung, Filterung, Rückprojektion
    \item Warum ist dieser Algorithmus besonders gut für Parallelisierung geeignet? -> viele viele Voxel, keine Abhängigkeiten
    \item Welche Algorithmen gäbe es noch?
\end{itemize}

\section{Parallelisierungsansätze}

\begin{itemize}
    \item Welche Ansätze sind seit 1984 (Erscheinungsjahr des FDK-Papers) entstanden? -> Vorstellung der Interessanten
    \item Welche Vor- und Nachteile haben diese Ansätze? -> Warum brauchen wir etwas neues / warum können wir sie nicht anwenden?
\end{itemize}


\section{Die CUDA-Plattform}

\begin{itemize}
    \item Warum GPUs für wissenschaftliche Berechnungen? -> Vergleich zu CPUs / anderen Beschleunigern
    \item Warum CUDA? Was gäbe es noch? (OpenCL, OpenMP, OpenACC, ...) / Nachteile?
    \item Das CUDA-Programmiermodell
\end{itemize}
