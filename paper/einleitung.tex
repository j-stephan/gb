\chapter{Einleitung}

\section{Die Geschichte und Relevanz der Computertomographie}

Die Geschichte der Computertomographie beginnt mit dem vom deutschen Physiker Wilhelm Conrad Röntgen entdeckten
und später nach ihm benannten Verfahren der {\glqq}X-Strahlen{\grqq} (vgl.~\cite{roentgen}). Plötzlich war es möglich,
die innere Beschaffenheit eines Objekts auf nichtinvasive Art und Weise zu untersuchen. Die Bedeutung dieses Verfahrens
insbesondere für die Anwendung in der Medizin war bereits Röntgens Zeitgenossen klar. So druckte die Wiener Zeitung
{\glqq}Die Presse{\grqq} am 05.\ Januar 1896 auf ihrer Titelseite unter der Überschrift {\glqq}Eine sensationelle
Entdeckung{\grqq}: {\glqq}[Man hat es] mit einem in seiner Art epochemachenden Ergebnisse der exacten Forschung zu thun,
das sowol [sic] auf physikalischem wie auf medicinischem Gebiete ganz merkwürdige Consequenzen bringen dürfte.{\grqq}
Für seine Entdeckung wurde Röntgen in der Folge unter anderem mit dem ersten Nobelpreis für Physik ausgezeichnet.

Führt man eine Röntgenaufnahme (\textit{Vorwärtsprojektion}) genügend oft in aufeinanderfolgenden Winkelschritten aus,
bis man (idealerweise) einen Vollkreis abgefahren hat, so lässt sich aus den dabei entstandenen Abbildungen
(\textit{Projektionen}) der durchleuchtete Körper rekonstruieren (\textit{Rückprojektion}). Diese Vorgehensweise wurde
in den 60er Jahren des 20. Jahrhunderts durch den südafrikanisch-amerikanischen Physiker Allan McLeod Cormack
entwickelt, der ebenfalls die dazu notwendigen mathematischen Grundlagen herleitete (vgl.~\cite{cormack63}
und~\cite{cormack64}). Ihm war allerdings nicht bekannt, dass diese Grundlagen schon 1917 vom österreichischen
Mathematiker Johann Radon gefunden wurden (vgl.~\cite{cormack79}). Mathematisch ist der Vorgang der
\textit{Rückprojektion} die Inversion der nach Radon benannten \textit{Radon-Transformation} (vgl.~\cite{radon}).

Da die Rückprojektion für jeden Punkt im Körper einzeln berechnet werden muss, ist sie für einen Menschen nicht in
sinnvoller Zeit lösbar. Aus diesem Grund ist man für die Lösung des Gesamtproblems auf einen Computer angewiesen, woraus
sich der Name des Verfahrens ableitet: \textit{Computertomographie}. Die ersten bis zur Marktreife entwickelten
Computertomographen wurden gegen Ende der 60er Jahre des 20. Jahrhunderts vom englischen Elektroingenieur Godfrey
Hounsfield gebaut (vgl.~\cite{houns}). Dieser entwickelte die für die Rückprojektion nötigen Algorithmen ebenfalls
selbst, da ihm die Vorarbeiten von Cormack und Radon nicht bekannt waren. Für ihre Arbeiten erhielten Hounsfield und
Cormack 1979 den Nobelpreis für Physiologie oder Medizin, was die Bedeutung der Computertomographie für die Medizin
unterstreicht.

Bis heute ist die Computertomographie ein bedeutendes Verfahren zur nichtinvasiven Analyse von Strukturen innerhalb
eines Körpers. Dabei wird sie nicht nur in der Medizin angewendet, sondern wird ebenfalls in der Materialforschung
eingesetzt. Für Projektionen, die im dreidimensionalen Raum mit Hilfe eines Kegelstrahls erzeugt wurden, haben Feldkamp,
Davis und Kress 1984 einen Rückprojektionsalgorithmus entwickelt, der einfach zu implementieren ist (vgl.~\cite{fdk}).
Aufgrund des technischen Fortschritts bei der Geräteentwicklung und Detektorauflösung und der damit zusammenhängenden
stetig wachsenden Datenmenge ist eine effiziente Implementierung der Rückprojektion auch heute noch wichtig. Eine
Möglichkeit zur effizienten Verarbeitung dieser Datenmenge liegt im Einsatz von Grafikkarten, deren für parallele
Berechnungen von computergrafischen Problemen ausgelegte Architektur für die gefilterte Rückprojektion große Vorteile
verspricht.

\section{Aufgabenstellung}\label{intro:aufgabenstellung}

Das Ziel dieser Arbeit ist, den \gls{fdk} mit Grafikkarten zu parallelisieren und diese Implementierung zu optimieren.
Zum Einsatz kommt dabei die vom Grafikkartenhersteller NVIDIA entwickelte \gls{cuda}-Plattform. Da der Einsatz von
Grafikkarten mit Limitierungen verbunden ist, wie etwa einem im Vergleich zu herkömmlichen Prozessoren begrenzten
Speicher, sollen die folgenden Implementierungsziele erreicht werden:

\begin{enumerate}
    \item Der begrenzte Speicher auf \gls{gpu}s soll möglichst effizient genutzt werden. Häufige Kopien zwischen dem
          Hauptspeicher der \gls{cpu} und dem der \gls{gpu} sind also zu vermeiden.
    \item Die Rückprojektion soll auf der \gls{gpu} möglichst optimal ausgeführt werden. Daraus folgt, dass
    \item die Rückprojektion die zur Verfügung stehenden \gls{gpu}s möglichst stark auslasten sollte, um keine
          Rechenressourcen zu verschwenden.
    \item Wenn mehrere \gls{gpu}s zur Verfügung stehen, sollen diese auch genutzt werden. Die Implementierung muss also
          skalierbar sein.
\end{enumerate}

Die Parallelisierung des \gls{fdk} ist seit seiner Vorstellung immer wieder Gegenstand der Forschung gewesen. Die
Umsetzung der vorgenannten Ziele bedingt also zunächst eine Analyse der bisher erfolgten Arbeiten, insbesondere im
Hinblick auf die Verwendung von \gls{gpu}s, sowie einen Variantenvergleich der erfolgsversprechenden Ideen. Aus diesem
ist dann ein Ansatz herzuleiten, dessen Implementierung dann im Hinblick auf die Zielstellung analysiert wird.
