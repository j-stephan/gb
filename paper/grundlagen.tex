\chapter{Grundlagen}

\section{Die Computertomographie}

\subsection{Die prinzipielle Funktionsweise der Computertomographie}

Am Anfang der Computertomographie steht das Röntgenverfahren, das 1895 vom deutschen Physiker Wilhelm Conrad Röntgen
entdeckt wurde~\cite{roentgen}. Mit Hilfe einer Strahlungsquelle wird ein Objekt durchleuchtet und auf einem Film bzw.\
einem Detektor abgebildet; der dreidimensionale Körper wird also auf eine zweidimensionale Fläche projiziert. Diesen
Schritt bezeichnet man als \textit{Vorwärtsprojektion}.

Führt man die Vorwärtsprojektion genügend oft in aufeinanderfolgenden Winkelschritten aus, bis man (idealerweise) einen
Vollkreis abgefahren hat, so lässt sich aus den dabei entstandenen \textit{Projektionen} der ursprünglich durchleuchtete
Körper, den wir in der Folge als \textit{Volumen} bezeichnen, rekonstruieren. Für jeden Punkt im Volumen
(\textit{Voxel}) kann anhand der Informationen aus den Projektionen der Absorptionsgrad berechnet und dadurch die
innere Struktur des Volumens bestimmt werden. Dieser Zusammenhang wurde in den 60er Jahren des 20. Jahrhunderts durch
den südafrikanisch-amerikanischen Physiker Allan McLeod Cormack festgestellt, der ebenfalls die dazu notwendigen
mathematischen Grundlagen entwickelte~\cite{cormack63}~\cite{cormack64}; ihm war allerdings unbekannt~\cite{cormack79},
dass diese schon 1917 vom österreichischen Mathematiker Johann Radon gefunden wurden~\cite{radon}. Mathematisch ist der
Vorgang der \textit{Rückprojektion} eine Anwendung der nach Radon benannten \textit{Radon-Transformation}.

Ein Problem der Vorwärtsprojektion ist der Informationsverlust, der durch die mangelnde Tiefe des Films bzw.\ Detektors
entsteht; die Tiefeninformationen werden auf die zweidimensionale Fläche {\glqq}verschmiert{\grqq}. Bei der
Rückprojektion lässt sich dieser Verlust durch die Wahl eines geeigneten Bildfilters wiederum kaschieren, weshalb man
auch von der \textit{gefilterten Rückprojektion} spricht.

Da die gefilterte Rückprojektion für jedes Voxel einzeln berechnet werden muss, ist sie für einen Menschen nicht in
sinnvoller Zeit lösbar. Aus diesem Grund ist man für die Lösung des Gesamtproblems auf einen Computer angewiesen, woraus
sich der Name des Verfahrens ableitet: \textit{Computertomographie}. Die ersten bis zur Marktreife entwickelten
Computertomographen wurden gegen Ende der 60er Jahre des 20. Jahrhunderts vom englischen Elektroingenieur Godfrey
Hounsfield gebaut. Dieser entwickelte die für die Rückprojektion nötigen Algorithmen ebenfalls selbst, da ihm die
Vorarbeiten von Cormack und Radon nicht bekannt waren~\cite{kalender}. Für ihre voneinander unabhängigen Arbeiten
erhielten Godfrey und Cormack 1979 den Nobelpreis für Physiologie oder Medizin, was die Bedeutung der
Computertomographie insbesondere für die Medizin unterstreicht.

\subsection{Der Feldkamp-Davis-Kress-Algorithmus}

Der 1984 entwickelte \gls{fdk}~\cite{fdk} ist eine spezielle Ausprägung der gefilterten Rückprojektion für die
Computertomographie mit Kegelstrahlen. Der Ausgangspunkt der Strahlung ist eine Quelle, die das Volumen mit einem
\textit{kegelförmigen} Strahl durchleuchtet und damit auf einem Detektor abbildet. Die so aufgenommenen
zweidimensionalen Projektionen werden dann pixelweise mit dem folgenden Wichtungsfaktor multipliziert:

\begin{equation}
    w_{ij} = \frac{d_{det} - d_{src}}{\sqrt{(d_{det} - d_{src})^2 + h_j^2 + v_i^2}}
\end{equation}

Zum Ausgleich der durch die Vorwärtsprojektion verloren gegangenen Tiefeninformationen werden die Projektionen im
nächsten Schritt zeilenweise gefiltert. Zu diesem Zweck müssen die Projektionen und der Filter allerdings mittels der
diskreten Fouriertransformation in den komplexen Raum transformiert werden; zum Einsatz kommt dabei das Verfahren der
schnellen Fouriertransformation (\textit{fast Fourier transform}, FFT) nach Cooley und Tukey~\cite{cooltuk}. Da dieses
Verfahren nur mit einer Menge von Elementen funktioniert, die einer Zweierpotenz entspricht, müssen die
Projektionszeilen und der Filter auf die nächste Zweierpotenz {\glqq}aufgerundet{\grqq} werden. Dazu wird, ausgehend von
der Länge einer Projektionszeile $N_h$,  die Filterlänge $N_{hFFT}$ berechnet:

\begin{equation}
    N_{hFFT} = 2 \cdot 2^{\left\lceil \frac{\log N_h}{\log 2} \right\rceil}
\end{equation}

Mit der so bestimmten Filterlänge lässt sich der Filter $r$ erzeugen:

\begin{equation}
    \begin{aligned}
        r(j) \text{ mit } j &\in \left[-\frac{N_{hFFT} - 2}{2}, \frac{N_{hFFT}}{2}\right]\\
        r(j) &=
            \begin{cases}
                \frac{1}{8} \cdot \frac{1}{\tau^2} & \quad \text{wenn } j = 0\\
                0 & \quad \text{wenn } j \text{ gerade}\\
                -\frac{1}{2j^2\pi^2\tau^2} & \quad \text{wenn } j \text{ ungerade}\\
            \end{cases}
    \end{aligned}
\end{equation}

Nun wird die zu filternde Zeile so lange mit $0$ aufgefüllt, bis die erweiterte Zeile $N_{hFFT}$ Pixel umfasst:

\begin{equation}
    \begin{aligned}
        p &: \text{ mit Nullen aufgefüllte Projektionszeile}\\
        p(0 \dots N_{h - 1}) &= \text{det}(0 \dots N_{h - 1})\\
        p(N_{h} \dots N_{hFFT}) &= 0
    \end{aligned}
\end{equation}

Im Anschluss werden sowohl der Filter $r$ als auch die erweiterte Projektionszeile $p$ in den komplexen Raum
transformiert und dort miteinander multipliziert:

\begin{equation}
    \begin{aligned}
        R &= \text{FFT}(r)\\
        P &= \text{FFT}(p)\\
        F &= P \cdot R \quad \text{sowohl für den reellen als auch den imaginären Teil}
    \end{aligned}
\end{equation}

Die so gefilterte Projektionszeile $F$ wird dann mit der inversen schnellen Fouriertransformation (IFFT) in den
reellen Raum zurücktransformiert und von den {\glqq}aufgefüllten{\grqq} Elementen bereinigt:

\begin{equation}
    \begin{aligned}
        f &= \text{IFFT}(F)\\
        \text{gefilterte Projektionszeile} &: f(0 \dots N_{h - 1})
    \end{aligned}
\end{equation}

Die auf diese Weise gefilterten Projektionen können nun wie folgt für die Rückprojektion verwendet werden:

Für jede Projektion $p$ mit dem Drehwinkel $\alpha_p$:

\begin{itemize}
    \item berechne für jede Voxelkoordinate $(x_k, y_l, z_m)$ deren rotierte Position $(s, t, z)$:
        \begin{equation}
            \begin{aligned}
                s &= x_k \cos \alpha_p + y_l \sin \alpha_p\\
                t &= -x_k \sin \alpha_p + y_l \cos \alpha_p\\
                z &= z_m
            \end{aligned}
        \end{equation}

    \item projiziere die rotierte Voxelkoordinate $(s, t, z)$ auf den Detektor:
        \begin{equation}
            \begin{aligned}
                h' &= y' = t \cdot \frac{d_{det} - d_{src}}{s - d_{src}}\\
                v' &= z' = z \cdot \frac{d_{det} - d_{src}}{s - d_{src}}
            \end{aligned}
        \end{equation}

    \item interpoliere das Detektorsignal bei $(h', v')$:
        \begin{equation}
            \begin{aligned}
                det' = det(h', v')
            \end{aligned}
        \end{equation}

    \item führe die Rückprojektion aus:
        \begin{equation}
            \begin{aligned}
                vol_{klm} &= vol_{klm} + 0,5 \cdot det' \cdot u^2\\
                \text{mit } u &= \frac{d_{src}}{s - d_{src}}
            \end{aligned}
        \end{equation}
\end{itemize}

Nach Abschluss der Rückprojektion erhält man ein Volumen, dessen Voxel Aufschluss über seine innere Struktur geben.

Aufgrund seiner geringen Komplexität und einfachen Implementierbarkeit ist der \gls{fdk} einer der beliebtesten
Rückprojektionsalgorithmen für die Kegelstrahl-Computertomographie~\cite{xumuell}. Der Vorteil des \gls{fdk} liegt
außerdem darin, dass die gefilterte Rückprojektion für jedes Voxel individuell berechnet werden kann, das heißt ohne
Abhängigkeiten zu anderen Voxeln. Dieser Umstand ermöglicht für die maschinelle Berechnung den maximalen Grad an
Parallelität, der im englischen Sprachraum auch als \textit{embarassingly parallel} bezeichnet wird, und macht den
\gls{fdk} zu einem idealen Ziel für diverse Parallelisierungsansätze. Einige neuere Ansätze sollen im Folgenden
vorgestellt werden.

\subsection{Bisherige Parallelisierungsansätze}

Seit seiner Einführung ist der \gls{fdk} ein beliebtes Untersuchungsobjekt diverser Forschungsgruppen, die sich mit
seiner Beschleunigung bzw.\ Parallelisierung mittels einer großen Variation von Architekturen, Plattformen und
Programmiermodellen beschäftigen. 

Xu et al.\ untersuchten bereits 2004, inwieweit sich der \gls{fdk} durch den Einsatz handelsüblicher Grafikkarten
(\textit{commodity graphics hardware}) beschleunigen lässt~\cite{xumuell}. Dabei wurden die Schritte \textit{Wichtung}
und \textit{Filterung} aufgrund ihrer geringen Komplexität ($\mathcal{O}(n^2)$) auf der CPU ausgeführt, während man die
komplexere \textit{Rückprojektion} ($\mathcal{O}(n^4)$) auf der GPU berechnete. Die Rückprojektion fand schichtweise
statt, jeweils für eine Voxelebene entlang der vertikalen Volumenachse. In ihrem Fazit stellten die Autoren die
Vermutung auf, dass der Abstand zwischen den Leistungen von CPU und GPU in der Zukunft zugunsten der GPUs immer weiter
werden würde: \textit{Since GPU performance has so far doubled every 6 months (i.e., triple of Moore's law), we expect
that the gap between CPU and GPU approaches will widen even further in the near future.}

Li et al.\ beschäftigten sich 2005 damit, wie man den \gls{fdk} mit einem \gls{fpga} implementieren könnte. Dazu teilten
sie das Ausgabevolumen, also die Zieldaten der Rückprojektion, in mehrere Würfel (\textit{bricks}) auf, um zu einer
optimalen Cachenutzung zu kommen. Der verwendete deterministische Aufteilungsalgorithmus hatte zur Folge, dass bei der
Berechnung auf dem \gls{fpga} kein Cache-Verfehlen (\textit{cache miss}) mehr auftrat.

Knaup et al.\ gingen 2007 der Frage nach, ob der \gls{fdk} durch die Eigenschaften der Cell-Architektur profitieren
könne~\cite{knaupsteck}.

Scherl et al.\ unternahmen 2008 den Versuch, den \gls{fdk} mittels \gls{cuda} zu beschleunigen~\cite{scherlkeck}. Im
Gegensatz zu der Gruppe um Xu et al.\ führten sie alle Schritte auf der GPU aus und führten die Rückprojektion
projektionsweise durch, das heißt, dass jede Projektion einzeln in das Gesamtvolumen zurückprojiziert wurde. Diese Art
der Datenverarbeitung ermöglichte es, die Schritte \textit{Wichtung} und \textit{Filterung} parallel zur Rückprojektion
auszuführen. Zur Ausnutzung dieser Eigenschaft und zur besseren Kapselung bzw.\ Modularisierung der Teilschritte
entwickelten die Autoren daher eine Pipeline-Struktur zur parallelen Abarbeitung des Algorithmus, basierend auf dem von
Mattson et al.\ vorgestellten Entwurfsmuster~\cite{mattsan}.

Balász et al.\ versuchten 2009 das Gleiche mit der \gls{opencl}~\cite{balgab}.

Hofmann et al.\ untersuchten eventuelle Vorteile durch den Einsatz der neuen Koprozessoren vom Typ 
Intel{\textregistered} Xeon Phi{\texttrademark} {\glq}Knights Corner{\grq}~\cite{hoftrei}.

Zhao et al.\ verfolgten die Absicht, eine Beschleunigung durch Ausnutzung geometrischer Zusammenhänge zu
erreichen~\cite{zhao}. Sie setzten dabei auf die Tatsache, dass ein einmal bestimmtes, also auf den Detektor
projiziertes, Voxel durch Rotation in 90°-Schritten die rotierten Voxel ebenfalls genau bestimmt. Ist also für ein
Voxel im Projektionswinkel 0° die zugehörige Detektorkoordinate gefunden, so kann diese Detektorkoordinate
für die Projektionswinkel 90°, 180° und 270° und die entsprechenden Voxel wiederverwendet werden.

\section{Die NVIDIA{\textregistered}-CUDA{\textregistered}-Plattform}

\subsection{Programmierbare Grafikkarten}

Als NVIDIA{\textregistered} im Jahre 2006 seine \textit{Compute-Unified-Device-Architecture}-Plattform
(CUDA{\textregistered}) vorstellte, die die direkte Programmierung der NVIDIA{\textregistered}-Grafikkarten ermöglichte,
folgte die Firma damit einer Entwicklung, die in den ersten Jahren des neuen Jahrtausends begonnen hatte. Durch die
Einführung von dezidierten Berechnungseinheiten für Gleitkommazahlen (\gls{fpu}) sowie den zunehmenden Funktionsumfang
der auf den Grafikkarten verbauten Shader-Einheiten wurde es theoretisch möglich, Berechnungen, die vorher nur von CPUs
ausgeführt werden konnten, nun auch von GPUs ausführen zu lassen. Dabei haben Grafikkarten gegenüber herkömmlichen
Prozessoren den Vorteil, dass die auf ihnen ausgeführten Berechnungen aufgrund ihrer für die Computergrafik optimierten
Bauweise -- also die Manipulation vieler Pixel zur gleichen Zeit -- automatisch \textit{datenparallel} sind. 
\textit{Datenparallelität} bezeichnet dabei die parallele Ausführung derselben Berechnung bzw.\ Anweisung auf
verschiedenen Daten. Dem gegenüber steht die \textit{Taskparallelität}, mit der eine parallele Abarbeitung verschiedener
Aufgaben gemeint ist. \textit{Taskparallelität} entspricht eher dem Programmiermodell der klassischen CPU, ist auf GPUs
dagegen nur begrenzt anwendbar.

Die \textit{datenparallele} Berechnung auf GPUs, meistens \gls{gpgpu} genannt, bietet sich insbesondere
für die Verarbeitung großer Datenmengen an, wie sie zum Beispiel in der Wissenschaft häufig vorkommt. Am Anfang der
2000er Jahre gab es allerdings keine komfortable Möglichkeit, die erhältlichen Grafikkarten direkt zu programmieren;
man war daher gezwungen, die zu berechnenden Daten zunächst in Objekte der Computergrafik umzuwandeln (beispielsweise 
Texturen) und mit den Mitteln der bestehenden Computergrafik-Bibliotheken wie der \gls{opengl} oder \gls{directx} zu
bearbeiten. Mit der Einführung von \gls{cuda} entfiel diese Beschränkung, da es nun möglich war, speziell für die
Grafikkarte geschriebene Programme auf dieser auszuführen, ohne den Umweg über Computergrafik-Bibliotheken gehen zu
müssen.

\subsection{Das CUDA{\textregistered}-Programmier- und Ausführungsmodell}

\subsection{Alternativen zu CUDA{\textregistered}}

Während \gls{cuda} ein großer Fortschritt im Bereich des \gls{gpgpu} war, so blieb es aufgrund der Tatsache, dass es
ein proprietäres Produkt von NVIDIA{\textregistered} ist, stets auf Grafikkarten und Beschleuniger dieses Herstellers
beschränkt. Der Computerhersteller Apple{\textregistered}, der in der zweiten Hälfte der 2000er Jahre Grafikkarten des
NVIDIA{\textregistered}-Konkurrenten AMD verbaute und seinen Kunden ebenfalls einen einfachen Zugang zu
\gls{gpgpu} ermöglichen wollte, entwickelte daher eine eigene, wenn auch stark an \gls{cuda} angelehnte,
\gls{gpgpu}-Plattform, die \gls{opencl} genannt wurde, und übergab diese kurz darauf dem Industriekonsortium Khronos zur
Standardisierung. Im Jahre 2008 wurde dann mit \gls{opencl} 1.0 die erste hardware- und herstellerunabhängige
\gls{gpgpu}-Plattform herausgegeben, die unter anderem von den drei größten GPU-Herstellern NVIDIA{\textregistered},
AMD und Intel{\textregistered} unterstützt wurde. Theoretisch konnte dasselbe Programm nun ohne Änderungen mit einer
beliebigen GPU beschleunigt werden. In der Praxis hinkte die Unterstützung des sukzessiver Versionen des Standards
jedoch der Entwicklung der Hardware hinterher, insbesondere in Bezug auf NVIDIA{\textregistered}-Hardware. Während der 
OpenCL 1.1 noch relativ zeitnah 

Parallel zu der Weiterentwicklung der Grafikkarten fand auch bei den herkömmlichen Prozessoren durch die allmähliche
Erhöhung der Kernanzahl eine Hinwendung zu parallelen Ausführungsmodellen statt. Ein Ansatz zur Vereinfachung der
Programmierung vieler CPU-Kerne ist das seit 1997 gemeinschaftlich entwickelte \gls{openmp}. Im Gegensatz zu den
Programmiermodellen von \gls{cuda} oder \gls{opencl}, die separate Programme auf den Beschleunigern starten, wird
\gls{openmp} direkt mit Compiler-Direktiven in das eigentliche Programm eingebettet.

An \gls{openmp} angelehnt ist \gls{openacc}, welches die \gls{gpgpu}-Programmierung mit Compiler-Direktiven
ermöglicht. Syntax und Funktionsweise ähneln stark der von \gls{openmp}, haben jedoch den Nachteil, dass eine auf
bestimmte Hardware zugeschnittene Optimierung nicht mehr so einfach möglich ist.
