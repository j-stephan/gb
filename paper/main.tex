\documentclass[utf8,german,beleg]{zihpub}

\usepackage{caption}
\usepackage{datetime}
\usepackage{etoolbox}
\usepackage[acronyms,nonumberlist,toc,xindy]{glossaries}
\usepackage[chapter,newfloat]{minted}
\usepackage{pgfplots}
\usepackage{tcolorbox}
\usepackage{tikz}

\newenvironment{code}{\captionsetup{type=listing}}{}
\SetupFloatingEnvironment{listing}{name=Quelltext}

\usetikzlibrary{arrows}

\BeforeBeginEnvironment{minted}{\begin{tcolorbox}}
\AfterEndEnvironment{minted}{\end{tcolorbox}}

\newcommand{\lstfont}[1]{\color{#1}\small\ttfamily}

\definecolor{keyword-green}{RGB}{0, 128, 0}

\author{Jan Stephan}
\title{Untersuchung der Parallelisierung des Feldkamp-Davis-Kress-Algorithmus mittels CUDA}
\matno{3755136}
\betreuer{\hspace{0.15mm}Dr.-Ing. André Bieberle \\
          \qquad \qquad Dr.-Ing. Guido Juckeland \\
          \qquad \qquad Matthias Werner}
\date{31. März 2017}
\copyrighterklaerung{Die in dieser Arbeit verwendeten Grafiken wurden -- soweit nicht anders angegeben -- von mir
                     persönlich entweder ohne Vorlage oder aber nach einer am jeweiligen Erscheinungsplatz zitierten
                     Vorlage erstellt. Für die von mir ohne Vorlage erstellten Grafiken behalte ich mir alle Rechte vor.
                     
                     Ich versichere ferner, diese Arbeit eigenständig angefertigt und aus anderen Werken übernommene
                     Zitate und Gedankengänge entsprechend kenntlich gemacht zu haben.}
\acknowledgments{Für die fachliche Betreuung bei der Erstellung dieser Arbeit bedanke ich mich recht herzlich bei Frau
                 Dr.-Ing.\ Martina Bieberle vom Helmholtz-Zentrum Dresden-Rossendorf sowie Herrn Dr.-Ing.\ Stephan Boden
                 von der AREVA-Stiftungsprofessur für bildgebende Messverfahren für die Energie- und Verfahrenstechnik.}
\bibfiles{bibliography.bib}

\newglossary{nomencl}{ncs}{nco}{Nomenklatur}

\newglossarystyle{nomenclstyle}{
    % Umgebung
    \renewenvironment{theglossary}
        {\begin{longtable}{llc}}
        {\end{longtable}}
    % Tabellenkopf
    \renewcommand*{\glossaryheader}{\bfseries Name & \bfseries Beschreibung & \bfseries Symbol \\\hline\endhead}
    % kein Abstand zwischen Gruppen
    \renewcommand*{\glsgroupheading}[1]{}
    \renewcommand*{\glossaryentryfield}[5]{
    \glstarget{##1}{##2}% Name
            & ##3 % Beschreibung
            & ##4 % Symbol
            \\ % Zeilenende
    }
    % Was zwischen den einzelnen Glossar-Gruppen zu tun ist
    \renewcommand*{\glsgroupskip}{\relax}
}

\makeglossaries

\newglossaryentry{kernel}{name=Kernel,
                          description={Programm, das auf einem Beschleuniger, wie etwa einer GPU, ausgeführt wird},
                          plural=Kernel}
\newglossaryentry{cuda}{name={CUDA},
                        description={NVIDIA CUDA, proprietäre Plattform für die Programmierung von Grafikkarten}}
\newglossaryentry{directx}{name={DirectX},
                           description={Microsoft DirectX, proprietäre Plattform für u.a. Grafikprogrammierung}}
\newglossaryentry{pixel}{name=Pixel,
                         description={Punkt in einem zweidimensionalen Koordinatensystem, z.B.\ einem Bild oder einem
                                      Detektor},
                         plural=Pixel}
\newglossaryentry{voxel}{name=Voxel,
                         description={Punkt in einem dreidimensionalen Koordinatensystem, z.B.\ einem Volumen},
                         plural=Voxel}
\newglossaryentry{host}{name=Host,
                        description={Gerät, das einen Kernel auf dem Device ausführt. Üblicherweise ist dies das Gerät,
                                     auf dem auch das Betriebssystem läuft, etwa ein Rechner oder ein Knoten auf einem
                                     Superrechner},
                        plural=Hosts}
\newglossaryentry{device}{name=Device,
                          description={Beschleuniger, der einen Kernel ausführt. Im Zusammenhang mit \gls{cuda} ist
                                       dieser typischerweise eine \gls{gpu}},
                          plural=Devices}
\newglossaryentry{stream}{name=Stream,
                          description={Warteschlange auf einem \gls{cuda}-\gls{device}. Operationen, wie z.B.\
                                       \gls{kernel}aufrufe, werden innerhalb eines Streams sequentiell ausgeführt.
                                       Mehrere Streams können vom gleichen \gls{device} parallel abgearbeitet werden},
                          plural=Streams}
\newglossaryentry{race-condition}{name=Race Condition,
                                  description={Paralleler, nicht (ausschließlich) lesender Zugriff mehrerer Threads auf
                                               das gleiche Datum. Da die Reihenfolge der Schreib- und Lesezugriffe ohne
                                               weitere Synchronisierungsmechanismen nicht definiert ist, besteht die
                                               Möglichkeit der Datenkorruption},
                                  plural=race conditions}
\newglossaryentry{grid}{name=Grid,
                        description={Menge aller auf dem \gls{device} durch einen \gls{kernel} gestarteten Threads},
                        plural=Grids}

\newacronym{fdk}{FDK-Algorithmus}{Feldkamp-Davis-Kress-Algorithmus}
\newacronym{hzdr}{HZDR}{Helmholtz-Zentrum Dresden-Rossendorf}
\newacronym{paris}{PARIS}{\textit{Portable and Accelerated 3D Reconstruction tool for radiation based Imaging
                                  Systems}}
\newacronym{fpga}{FPGA}{\textit{Field Programmable Gate Array}}
\newacronym{opencl}{OpenCL}{\textit{Open Computing Language}}
\newacronym{fpu}{FPU}{\textit{floating point unit}}
\newacronym{gpgpu}{GPGPU}{\textit{General Purpose Computation on Graphics Processing Unit}}
\newacronym{opengl}{OpenGL}{\textit{Open Graphics Library}}
\newacronym{openmp}{OpenMP}{\textit{Open Multi-Processing}}
\newacronym{openacc}{OpenACC}{\textit{Open Accelerators}}
\newacronym{gpu}{GPU}{\textit{Graphics Processing Unit}}
\newacronym{cpu}{CPU}{\textit{Central Processing Unit}}
\newacronym{simd}{SIMD}{\textit{Single Instruction, Multiple Data}}
\newacronym{simt}{SIMT}{\textit{Single Instruction, Multiple Threads}}
\newacronym{sm}{SM}{\textit{Streaming Multiprocessor}}
\newacronym{gpc}{GPC}{\textit{Graphics Processing Cluster}}
\newacronym{cufft}{cuFFT}{NVIDIA \gls{cuda} \textit{Fast Fourier Transform}}
\newacronym{hpc}{HPC}{\textit{High Performance Computing}}

\newglossaryentry{proj}{name=Projektion,
                        description={Röntgenaufnahme eines Objekts},
                        symbol={$p$},
                        plural=Projektionen,
                        type=nomencl}
\newglossaryentry{fourproj}{name={fouriertransformierte Projektion},
                            description={in den Frequenzraum überführte Projektion},
                            symbol={$P$},
                            plural={fouriertransformierte Projektionen},
                            sort={Projektion, fouriertransformiert},
                            type=nomencl}
\newglossaryentry{filtproj}{name={gefilterte Projektion},
                            description={im Frequenzraum gewichtete Projektion},
                            symbol={$p_F$},
                            plural={gefilterte Projektionen},
                            sort={Projektion, gefiltert},
                            type=nomencl}
\newglossaryentry{obj}{name=Objekt,
                       description={von Röntgenstrahlen durchleuchtetes Objekt},
                       symbol={$f$},
                       plural=Objekte,
                       type=nomencl}
\newglossaryentry{fourobj}{name={fouriertransformiertes Objekt},
                           description={in den Frequenzraum überführtes Objekt},
                           symbol={$F$},
                           plural={fouriertransformierte Objekte},
                           sort={Objekt, fouriertransformiert},
                           type=nomencl}
\newglossaryentry{vol}{name=Volumen,
                       description={aus Projektionen rekonstruiertes Objekt},
                       symbol={$v$},
                       plural=Volumina,
                       type=nomencl}
\newglossaryentry{dsrc}{name=Quelle-Objekt-Abstand,
                        description={Abstand zwischen Quelle und Objektmittelpunkt},
                        symbol={$d_{src}$},
                        sort={Abstand, Quelle-Objekt},
                        type=nomencl}
\newglossaryentry{ddet}{name=Objekt-Detektor-Abstand,
                        description={Abstand zwischen Objektmittelpunkt und Detektor},
                        symbol={$d_{det}$},
                        sort={Abstand, Objekt-Detektor},
                        type=nomencl}
\newglossaryentry{dsd}{name=Quelle-Detektor-Abstand,
                       description={Abstand zwischen Quelle und Detektor},
                       symbol={$d_{sd}$},
                       sort={Abstand, Quelle-Detektor},
                       type=nomencl}
\newglossaryentry{nh}{name=Zeilenlänge,
                      description={Pixelzahl pro Projektionszeile},
                      symbol={$N_h$},
                      sort={Länge, Zeile},
                      type=nomencl}
\newglossaryentry{nv}{name=Spaltenlänge,
                      description={Pixelzahl pro Projektionsspalte},
                      symbol={$N_v$},
                      sort={Länge, Spalte},
                      type=nomencl}
\newglossaryentry{dh}{name={Pixelbreite},
                      description={physische Breite eines Detektorpixels},
                      symbol={$d_h$},
                      sort={Pixelabstand, horizontal},
                      type=nomencl}
\newglossaryentry{dv}{name={Pixelhöhe},
                      description={physische Höhe eines Detektorpixels},
                      symbol={$d_v$},
                      sort={Pixelabstand, vertikal},
                      type=nomencl}

\glsadd{filtproj}
\glsadd{vol}
\glsadd{dsd}
\glsadd{nh}
\glsadd{nv}
\glsadd{dh}
\glsadd{dv}


% Germanisiere LaTeX
\deftranslation[to=German]{Acronyms}{Abkürzungsverzeichnis}
\deftranslation[to=German]{Glossary}{Glossar}
\renewcommand*{\listlistingname}{Quelltextverzeichnis}

\begin{document}

\printglossaries
\listoffigures
\listoflistings
\listoftables

\chapter{Einleitung}

\section{Die Geschichte und Relevanz der Computertomographie}

Die Geschichte der Computertomographie beginnt mit dem vom deutschen Physiker Wilhelm Conrad Röntgen entdeckten
und später nach ihm benannten Verfahren der {\glqq}X-Strahlen{\grqq}~\cite{roentgen}. Es war nun möglich, die innere
Beschaffenheit eines Objekts auf nichtinvasive Art und Weise -- also ohne es zu beschädigen -- festzustellen. Die
Bedeutung dieses Verfahrens insbesondere für die Anwendung in der Medizin war bereits Röntgens Zeitgenossen klar.
So druckte die Wiener Zeitung {\glqq}Die Presse{\grqq} am 05.\ Januar 1896 auf ihrer Titelseite unter der Überschrift
{\glqq}Eine sensationelle Entdeckung{\grqq}: {\glqq}[Man hat es] mit einem in seiner Art epochemachenden Ergebnisse der
exacten Forschung zu thun, das sowol[sic] auf physikalischem wie auf medicinischem Gebiete ganz merkwürdige Consequenzen
bringen dürfte.{\grqq} Für seine Entdeckung wurde Röntgen in der Folge unter anderem mit dem ersten Nobelpreis für
Physik ausgezeichnet. Bis heute ist das Röntgenverfahren ein wichtiger Bestandteil der medizinischen Diagnostik und der
Werkstoffprüfung.

\section{Der Einsatz der Computertomographie am Helmholtz-Zentrum Dresden-Rossendorf}

\begin{itemize}
    \item Ausgangssituation am HZDR:\@ Uraltes FDK-Programm braucht mehrere Tage für eine Rekonstruktion (schlecht)
    \item Gesamtziel: Dieses Programm soll durch ein schnelleres und möglichst gut optimiertes abgelöst werden
\end{itemize}

\section{Aufgabenstellung}

Der \gls{fdk} ist ein weit verbreiteter Ansatz zur Rekonstruktion von kegelförmiger Computer-Tomographie. In diesem
Beleg soll untersucht werden:

\begin{itemize}
    \item Zusammenfassung des Forschungsstandes hinsichtlich der Parallelisierung / der Verwendung von \gls{cuda}
    \item Gegenüberstellung verschiedener Optimierungsziele (Time-to-solution, Occupancy)
    \item Variantenvergleich verschiedener Implementierungsstrategien
    \item Implementierung und Analyse einer dieser Strategien
\end{itemize}

\chapter{Grundlagen der Computertomographie}\label{chap:grundlagen_bp}

Dieses Kapitel erläutert die für diese Arbeit relevanten theoretischen Grundlagen der Computertomographie. Es werden
zunächst die mathematischen Voraussetzungen gezeigt, darauf aufbauend schließt sich die Beschreibung des \gls{fdk} an.

\section{Mathematische Grundlagen}

In diesem Abschnitt werden die mathematischen Grundlagen der Computertomographie behandelt. Es werden zunächst die
mathematischen Eigenschaften der Vorwärtsprojektion und das sich daraus ergebende Fourier-Schichten-Theorem erläutert.
Anschließend wird das Prinzip der daraus abgeleiteten gefilterte Rückprojektion erläutert.

\subsection{Projektionen}

Schießt man einen Röntgenstrahl durch ein festes Objekt, wie beispielsweise biologisches Gewebe oder ein Metall, so wird
dieser Strahl je nach Dichte des Materials entlang seiner Bahn abgeschwächt bzw.\ absorbiert. Mathematisch lässt sich
ein Objekt daher als zwei- oder dreidimensionale Verteilung von Absorptionskonstanten verstehen, während die gesamte
Abschwächung entlang einer Strahlbahn als Kurvenintegral dargestellt werden kann.

Die Grundlage der folgenden Ausführungen ist die Abbildung~\ref{fig:math_proj}. Als Beispiel dienen ein Objekt, hier
durch die Funktion $f(x, y)$ repräsentiert, sowie Kurvenintegrale, hier das Parameterpaar $(\alpha, t)$. Die Linie $AB$
lässt sich dann durch die folgende Formel darstellen:

\begin{equation*}
    x \cdot \cos \theta + y \cdot \sin \theta  = t_1
\end{equation*}

oder allgemein für beliebige, zu $AB$ parallele, Linien:

\begin{equation}\label{eq:proj_obj}
    x \cdot \cos \theta + y \cdot \sin \theta = t
\end{equation}

\begin{figure}[!htb]
\centering
\begin{tikzpicture}[
    scale=0.8,
    axis/.style={thick,->}]
    % Achsen
    \draw[axis] (-5, 0) -- (8, 0) node[right] {$x$};
    \draw[axis] (0, -5) -- (0, 5) node[above] {$y$};
    \draw[axis] (-5, 7) -- (8, 9.6) node[right,sloped] {$t$};

    % Objekt
    \draw[fill=black!60!white,opacity=0.3] (0, -3) .. controls (3.5, -2.5) and (5, -2) .. (7, 0)
                                            .. controls (7.5, 1) and (6.5, 2.5) .. (6, 2.5)
                                            .. controls (4, 3) and (1, 2.5) .. (0,3)
                                            .. controls (-2, 3.2) and (-2.15, 2.4) .. (-2.75, 2)
                                            .. controls (-3.9, 0.5) .. (-4, 0)
                                            .. controls (-3.5, -2.5)  and (-1, -3.25) .. (0, -3);

    % Strahlen
    \draw[->] (3, -7) -- (0, 8) node[pos=0.9,above,sloped] {$A$} node[pos=0.2,above,sloped] {Strahl}
              node[pos=0.2,below,sloped] {$x \cdot \cos \theta + y \cdot \sin \theta = t_1$}
              node[pos=0,right,sloped] {$B$};

    \draw[->] (6, -6.4) -- (3, 8.6) node[pos=0.2,above,sloped] {Strahl}
              node[pos=0.2,below,sloped] {$x \cdot \cos \theta + y \cdot \sin \theta = t$};

    % t
    \draw[<->] (-0.3, 1.5) -- (1.25, 1.81) node[pos=0.5,above,sloped] {$t_1$};
    \draw[<->] (-0.15, 0.75) -- (4.4, 1.66) node[pos=0.5,above,sloped] {$t$};

    % sonstiges
    \draw[dashed] (-5, -1) -- (8, 1.6) node[pos=0,sloped,left] {$c$} node[sloped,right] {$c'$};
    \draw[dashed] (0.4, -2) -- (-0.4, 2);

    % Projektion
    \draw[<->] (0, 8) -- (-0.4, 10) node[pos=0.5, sloped, above] {$P_{\theta}(t)$};
    \draw[<->] (3, 8.6) -- (2.8, 9.6);
    \draw[thick] (-4, 7.2) .. controls (-3.8, 7.5) .. (-3, 8)
                 .. controls (-2.75, 8.5) and (-1.5, 9.25) .. (-1, 9.5)
                 .. controls (-0.6, 10.05) .. (-0.4, 10)
                 .. controls (1.2, 9.5) .. (2.3, 9.7)
                 .. controls (2.8, 9.6) .. (6, 9.5)
                 .. controls (6.25, 9.3) .. (7.5, 9.5);

    % Beschriftungen
    \draw[->] (-4.75, 2.5) -- (-2.75, 2) node[pos=0,left] {$f(x, y)$};
    \draw[->] (-4, 10.5) -- (-1, 9.5) node[pos=0, left] {Projektion};

    % Winkel
    \draw (2.5, 0) arc (0:23.5:12mm) node[pos=0.5,left] {$\theta$};
\end{tikzpicture}
\caption{Zusammenhang zwischen Kurvenintegral und Projektion (Vorlage:~\cite{kak79})}
\label{fig:math_proj}
\end{figure}

Das zu $f(x, y)$ gehörige Kurvenintegral ist $P_{\theta}(t)$:

\begin{equation}\label{eq:proj_int}
    P_{\theta}(t) = \int\limits_{(\theta, t)\text{-Linie}} f(x, y)\ \mathrm{d}s
\end{equation}

Zusammen mit der aus Formel~\ref{eq:proj_obj} resultierenden Delta-Distribution lässt sich das Kurvenintegral wie folgt
umschreiben:

\begin{equation}\label{eq:proj_radon}
    P_{\theta}(t) = \int\limits_{-\infty}^{\infty}\int\limits_{-\infty}^{\infty}f(x, y) \cdot \delta(x \cdot
                    \cos \theta + y \cdot \sin \theta - t)\ \mathrm{d} x\ \mathrm{d} y
\end{equation}

Die Funktion $P_{\theta}(t)$ ist die \textit{Radon-Transformation} der Funktion $f(x, y)$ (vgl.~\cite{radon}).

Eine Projektion lässt sich als die Kombination einer Menge von Radon-Transformationen verstehen. Die
(mathematisch) einfachste Projektion ist eine Sammlung von Parallelstrahlintegralen $P_{\theta}(t)$ unter einem
konstanten Winkel $\theta$. Man bezeichnet eine solche Projektion als \textit{Parallelstrahlprojektion} (siehe
Abbildung~\ref{fig:par_proj}). In der Praxis kann eine Parallelstrahlprojektion durch die Bewegung einer
Quelle-Detektor-Anordnung entlang paralleler Linien auf entgegengesetzten Seiten des Objekts aufgenommen werden.

Eine zweite Aufnahmemöglichkeit ist der Einsatz einer Quelle auf einer festen Position sowie einer Reihe von Detektoren
entlang einer Linie auf der anderen Seite des Objekts (siehe Abbildung~\ref{fig:fan_proj}). Solcherart erzeugte
Projektionen nennt man aufgrund der fächerförmigen Strahlen \textit{Fächerstrahlprojektionen} (vgl.~\cite{kakslan},
S. 49 -- 51).

\begin{figure}
    \centering
    \begin{tikzpicture}[axis/.style={thick,->}]
        \draw[axis] (-5, 0) -- (5, 0) node [right] {$x$};
        \draw[axis] (0, -5) -- (0, 5) node [right] {$y$};

        % Objekt
        \draw[thick] (1, 0) to [bend left] (1, 0.3) to [bend right] (0.4, 1.4) to (-0.3, 1.3)
                     to [bend left] (-0.85, 0.9) to [bend right] (-1, 0.3) to [bend left] (-1.2, 0)
                     to [bend right] (-1, -0.9) to [bend right] (0, -1.2) to (0.3, -1.3) to [bend right] (1.1 ,-0.7)
                     to [bend left] (1, 0);
        \draw[->] (1.5, 2.5) -- (0.6, 1.3) node [pos=0, above right] {$f(x, y)$};

        % Detektorebenen
        \draw[dashed] (-2.5, -2.2) -- (2.5, 2.2);
        \draw[dashed] (-2, -0.4) -- (4, 0.8);

        % Detektor 1
        \draw (-5.7, 4.05) -- (-3.55, 6);
        \draw[axis] (-4.95, 5) -- (-4.4, 5.5) node [sloped, above right] {$t$};
        \draw[thick] (-5.65, 4.1) to [bend left] (-5.5, 4.7) to [bend right] (-5.55, 4.85)
                                  to [bend left] (-5.4, 5.3) to [bend right] (-5.35, 5.7)
                                  to [bend left] (-5.1, 5.8) to [bend right] (-4.4, 5.95)
                                  to [bend left] (-4.2, 6) to [bend right] (-3.85, 6)
                                  to [bend left] (-3.7, 5.9);
        \node at (-5.25, 6.5) {$P_{\theta_1}(t)$};

        % Strahlen zu Detektor 1
        \foreach \w in {0,...,12}
            \draw [->] (1.8 + \w * 0.16666666666667, -4 + \w * 0.14583333333333)
                        -- (-5.55 + \w * 0.16666666666667, 4 + \w * 0.15);

        % Detektor 2
        \draw (-2.6, 6.2) -- (0, 6.72);
        \draw[axis] (-1.55, 6.55) -- (-0.75, 6.7) node [above right] {$t$};
        \draw[thick] (-2.45, 6.25) to [bend left] (-2.35, 6.6) to [bend right] (-2.2, 6.8)
                                   to [bend left] (-2.05, 7.35) to [bend right] (-1.5, 7)
                                   to [bend left] (-0.85, 7.4) to [bend left] (-0.35, 6.85)
                                   to [bend right] (-0.15, 6.7);
        \node at (-1.75, 8) {$P_{\theta_2}(t)$};

        % Strahlen zu Detektor 2
        \foreach \w in {0,...,12}
            \draw [->] (-0.2 + \w * 0.1833333333333, -4.75 + \w * 0.0375)
                        -- (-2.5 + \w * 0.1833333333333, 6.1 + \w * 0.0375);

        % Winkel
        \draw [->] (3.1, 0) arc (0:42:30mm) node [pos=0.6, fill=white] {$\theta_1$};
        \draw [->] (3.8, 0) arc (0:11:38mm) node [pos=0.5, right] {$\theta_2$};
    \end{tikzpicture}
    \caption{Parallelstrahlprojektionen (Vorlage:~\cite{rosenkak}, S. 356)}
    \label{fig:par_proj}
\end{figure}


\begin{figure}
    \centering
    \begin{tikzpicture}[axis/.style={thick,->}]
        \draw[axis] (-5, 0) -- (5, 0) node [right] {$x$};
        \draw[axis] (0, -5) -- (0, 5) node [above] {$y$};

        % Objekt
        \draw[thick] (0.9, 0) to [bend left] (0.8, 0.7) to [bend right] (0.7, 0.9) to [bend right] (-0.5, 1)
                              to [bend right] (-0.9, 0.7) to [bend left] (-1.2, 0.5)
                              to [bend right] (-1.2, -0.3) to [bend left] (-0.75, -1) to (-0.3, -1.2)
                              to [bend right] (0.2, -1.1) to [bend left] (1, -0.7) to [bend right] (0.9, 0.0);
        \draw[->] (2.7, 1.3) -- (0.7, 0.9) node [pos=0, right] {$f(x, y)$};

        % Quellen
        \coordinate (s1) at (4, -3.55);
        \coordinate (s2) at (-1.3, -5.3);
        \draw[fill=black] (s1) circle (2pt) node [right] {$s_1$};
        \draw[fill=black] (s2) circle (2pt) node [left] {$s_2$};

        % Detektor 1
        \draw (-6.1, 2.1) -- (-2.7, 5.8);
        \draw[axis] (-4.9, 3.7) -- (-4.45 , 4.25) node [above right] {$t$};
        \draw[thick] (-6.1, 2.1) to [bend left] (-6.05, 2.5) to [bend left] (-6, 3.5) to [bend right] (-6, 4.2)
                                 to [bend left] (-5.7, 4.9) to [bend right] (-4.9, 5.7) to [bend left] (-4.4, 5.8)
                                 to (-2.7, 5.8);
        \node at (-4.5, 6.5) {$P(t, \beta_1)$};

        % Strahlen zu Detektor 1
        \foreach \w in {0,...,12}
            \draw[->] (s1) -- (-5.9 + \w * 0.266666666666, 2.05 + \w * 0.295833333333333);

        % Detektor 2
        \draw (-1.05, 6.2) -- (3.7, 5.05);
        \draw[axis] (0.9, 5.9) -- (1.75, 5.7) node [right] {$t$};
        \draw[thick] (-1.05, 6.2) to [bend left] (-0.3, 6.4) to [bend right] (1, 6.5) to [bend left] (1.5, 6.6)
                                  to [bend right] (2.1, 6.8) to [bend left] (2.35, 6.8) to [bend left] (2.8, 6.3)
                                  to [bend right] (3.2, 5.6) to [bend left] (3.7, 5.05);
        \node at(1.75, 7.5) {$P(t, \beta_2)$};

        % Strahlen zu Detektor 2
        \foreach \w in {0,...,13}
            \draw[->] (s2) -- (-1.1 + \w * 0.36538415384, 6 - \w * 0.084615384);

        % Winkel
        \draw[->] (0, -2.2) arc (270:317.5:22.5mm) node[pos=0.3,below] {$\beta_1$};
        \draw[->] (0, -4) arc (270:251.5:30mm) node[pos=0.3,below] {$\beta_2$};
    \end{tikzpicture}
    \caption{Fächerstrahlprojektionen (Vorlage:~\cite{rosenkak}, S. 357)}
    \label{fig:fan_proj}
\end{figure}

\subsection{Das Fourier-Scheiben-Theorem}\label{sssec:fourier_scheibe}

Der Zusammenhang zwischen der Projektion und dem Objekt wird noch deutlicher, wenn man erstere einer eindimensionalen
und letztere einer zweidimensionalen Fouriertransformation unterzieht. So stellten Rosenfeld und Kak in ihrer Arbeit
fest, dass die Fouriertransformation einer Parallelstrahlprojektion eines Objekts $f(x, y)$, die unter dem Winkel
$\theta$ aufgenommen wurde, einer Schicht des zweidimensional fouriertransformierten Objekts $F(u, v)$, entspricht. Mit
anderen Worten ergibt die Fouriertransformation einer Projektion $P_{\theta}(t)$ die Werte von $F(u, v$) entlang einer
radialen Linie $BB$ (siehe Abbildung~\ref{fig:fourier_scheibe}, vgl.~\cite{rosenkak}, S.366).

Rosenfeld und Kak haben diesen Zusammenhang auch mathematisch hergeleitet. Seien $F(u, v)$ die zweidimensionale
Fouriertransformation des Objekts $f(x, y)$:

\begin{equation}\label{eq:obj_2d_full}
    F(u, v) = \int\limits_{-\infty}^{\infty} \int\limits_{-\infty}^{\infty} f(x, y) \cdot e^{-2 \pi i \cdot(u x + v y)}\
        \mathrm{d} x\ \mathrm{d} y
\end{equation}

und $S_{\theta}(w)$ die eindimensionale Fouriertransformation der Projektion unter dem Winkel $\theta$ $P_{\theta}(t)$:

\begin{equation}\label{eq:proj_fft}
    S_{\theta}(w) = \int\limits_{-\infty}^{\infty} P_{\theta}(t) \cdot e^{-2 \pi i \cdot w t}\ \mathrm{d} t
\end{equation}

Im Folgenden sei der Fall $\theta = 0$ betrachtet. Setzt man $v = 0$, so wird die Formel~\ref{eq:obj_2d_full}
vereinfacht:

\begin{equation}
    F(u, 0) = \int\limits_{-\infty}^{\infty} \int\limits_{-\infty}^{\infty} f(x, y) \cdot e^{-2 \pi i \cdot u x}\
    \mathrm{d} x\ \mathrm{d} y
\end{equation}

Der Phasenfaktor ist nun nicht mehr von $y$ abhängig, wodurch das Integral in zwei Hälften geteilt werden kann:

\begin{equation}\label{eq:obj_2d_split}
    F(u, 0) = \int\limits_{-\infty}^{\infty} \left[\int\limits_{-\infty}^{\infty} f(x, y)\ \mathrm{d} y \right] \cdot
    e^{-2 \pi i \cdot u x}\ \mathrm{d} x
\end{equation}

Der Term in eckigen Klammern entspricht der Gleichung für eine Projektion entlang konstanter $x$-Linien:

\begin{equation}
    P_{\theta = 0}(x) = \int\limits_{-\infty}^{\infty} f(x, y)\ \mathrm{d} y
\end{equation}

Durch Einsetzen in Gleichung~\ref{eq:obj_2d_split} erhält man:

\begin{equation}
    F(u, 0) = \int\limits_{-\infty}^{\infty} P_{\theta = 0}(x) \cdot e^{-2 \pi i \cdot u x}\ \mathrm{d} x
\end{equation}

Die rechte Seite dieser Gleichung stellt die eindimensional fouriertransformierte Projektion $P_{\theta = 0}$ dar (vgl.
Gleichung~\ref{eq:proj_fft}), es ergibt sich also der folgende Zusammenhang zwischen der transformierten Projektion und
dem zweidimensional fouriertransformierten Objekt:

\begin{equation}
    F(u, 0) = S_{\theta = 0}(u)
\end{equation}

Dieses Ergebnis ist die einfachste Form des Fourier-Scheiben-Theorems. Darüber hinaus ist es unabhängig von der
konkreten Konstellation zwischen dem Objekt und dem Koordinatensystem der Projektion. Wird das
$(t, s)$-Koordinatensystem um den Winkel $\theta$ rotiert, so ist die Fouriertransformation der Projektion gleich der
zweidimensionalen Fouriertransformation des Objekts entlang einer radialen Linie, die um den Winkel $\theta$ rotiert
wird (siehe Abbildung~\ref{fig:fourier_scheibe}, vgl.~\cite{rosenkak}, S. 366).

Aus den obigen Ergebnissen folgt, dass sich die Werte von $F(u, v)$ entlang radialer Linien dadurch bestimmen lassen,
dass man die Projektionen des Objekts unter mehreren Winkeln $\theta_1$, $\theta_2$, \ldots, $\theta_k$ aufnimmt und
diese fouriertransformiert. Für eine unendliche Anzahl von Projektionen wäre $F(u, v)$ somit in allen Punkten der
$uv$-Ebene bestimmt, sodass die Objektfunktion $f(x, y)$ durch die inverse Fouriertransformation bestimmt werden kann.
In der Praxis können allerdings nur endlich viele Projektionen aufgenommen werden, sodass $F(u, v)$ ebenfalls nur
entlang einer endlichen Anzahl radialer Linien bestimmt ist (siehe Abbildung~\ref{fig:fourier_scheibe_endlich}). Um
$f(x, y)$ dennoch näherungsweise bestimmen zu können, müssen die restlichen Punkte anhand der bekannten Werte
interpoliert werden, was typischerweise durch eine \textit{nearest neighbour} oder eine lineare Interpolation geschieht.
Da die Dichte der vorhandenen Punkte immer geringer wird, je weiter man sich vom Ursprung entfernt, nimmt der
Interpolationsfehler bei größerer Distanz zu. Die Komponenten eines Bildes mit hoher Frequenz sind somit stärker von
Fehlern betroffen als die niederfrequenten Anteile, was einen Qualitätsverlust des rekonstruierten Bildes zur Folge hat
(vgl.~\cite{kakslan}, S. 59 -- 60).

\begin{figure}
    \centering
    \begin{tikzpicture}[axis/.style={thick,->}]
        % links
        \draw[axis] (-5.5, 0) -- (-1, 0) node [right] {$x$};
        \draw[axis] (-3.25, -2.5) -- (-3.25, 2.5) node[pos=0,below] {Ortsraum} node [above] {$y$};

        % Objekt
        \draw[thick,fill=black!60!white,opacity=0.3] (-2.35, 0) to [bend right] (-2.45, 0.5) to [bend right] (-2.65, 1)
                                                                to [bend right] (-3.35, 1) to [bend right] (-3.85, 0.8)
                                                                to [bend left] (-4.25, 0.35) to [bend right] (-4.65, 0)
                                                                to [bend right] (-4.15 ,-0.35)
                                                                to [bend right] (-3.75, -1.1)
                                                                to [bend right] (-2.75, -0.9)
                                                                to [bend right] (-2.55 ,-0.5)
                                                                to [bend left] (-2.35, 0);
        \draw[->] (-4.65, -0.95) -- (-4.15, -0.65) node[pos=0, left] {Objekt};

        % Detektor
        \draw[axis] (-6.05, 1.475) -- (-3.25, 3.3) node[pos=1,sloped,right] {$t$};
        \draw[thick] (-5.7, 1.7) to [bend left] (-5.6, 2.25) to [bend right] (-5.35, 2.6)
        to [bend left] (-5.05, 2.75) to [bend left] (-4.5, 3.15)
        to [bend left] (-3.9733333333333, 2.82);

        % Pfeile zum Detektor
        \foreach \w in {0,...,6}
            \draw[->] (-3.25 + \w * 0.1233333333333, -1.95 + \w * 0.08)
                      -- (-5.7 + \w * 0.1233333333333, 1.7 + \w * 0.08);
        \draw[->] (-3.25 + 7 * 0.1233333333333, -1.95 + 7 * 0.08) -- (-5.5, 3.3);
        \foreach \w in {8,...,14}
            \draw[->] (-3.25 + \w * 0.1233333333333, -1.95 + \w * 0.08)
                      -- (-5.7 + \w * 0.1233333333333, 1.7 + \w * 0.08);

        % Linie
        \draw[dashed] (-3.25, 0) -- (-1.5, 1.15);

        % Winkel
        \draw[->] (-1.55, 0) arc(0:32:17.5mm) node[pos=0.5,right] {$\theta$};

        % rechts
        \draw[axis] (1, 0) -- (5.5, 0) node [right] {$u$};
        \draw[axis] (3.25, -2.5) -- (3.25, 2.5) node [pos=0,below] {Frequenzraum} node [above] {$v$};

        % Linie
        \draw (1.15, -1.4) -- (5.2, 1.3) node[sloped,pos=0,left] {$B$} node[sloped,pos=1,right] {$B$};

        % Winkel
        \draw[->] (4.6, 0) arc (0:32:14.5mm) node[pos=0.5,right] {$\theta$};

        % Verbindung
        \draw[<->,dotted,thick] (-4.5, 3.15) -- (-4.15, 4.2) -- (-1.45, 4.2) -- (1.75, -0.95);
        \node[draw,fill=white] at (-0.25, 3) {Fouriertransformation};
    \end{tikzpicture}
    \caption{Das Fourier-Scheiben-Theorem (Vorlage:~\cite{kakslan}, S. 57)}
    \label{fig:fourier_scheibe}
\end{figure}

\begin{figure}
    \centering
    \begin{tikzpicture}[axis/.style={thick,->}]
        \draw[axis] (-3, 0) -- (3, 0) node [right] {$u$};
        \draw[axis] (0, -3) -- (0, 3) node[pos=0,below] {Frequenzraum} node [above] {$v$};

        % Kreise
        \draw[dashed] (0, 0) circle (3mm);
        \draw[dashed] (0, 0) circle (6mm);
        \draw[dashed] (0, 0) circle (9mm);
        \draw[dashed] (0, 0) circle (12mm);
        \draw[dashed] (0, 0) circle (15mm);
        \draw[dashed] (0, 0) circle (18mm);

        % Linien
        \draw[thick] (-1.55, -0.9) -- (1.55, 0.9);
        \draw[thick] (-0.9, -1.55) -- (0.9, 1.55);
        \draw[thick] (0.9, -1.55) -- (-0.9, 1.55);
        \draw[thick] (1.55, -0.9) -- (-1.55, 0.9);

        % Punkte
        \foreach \w in {0,...,6}
            \draw[fill=black] (0, 0 + \w * 0.3) circle (1pt);
        \foreach \w in {0,...,6}
            \draw[fill=black] (0, 0 - \w * 0.3) circle (1pt);

        \foreach \w in {0,...,6}
            \draw[fill=black] (0 + \w * 0.3, 0) circle (1pt);
        \foreach \w in {0,...,6}
            \draw[fill=black] (0 - \w * 0.3, 0) circle (1pt);

        \foreach \w in {0,...,6}
            \draw[fill=black] (30:\w * 3mm) circle (1pt);
        \foreach \w in {0,...,6}
            \draw[fill=black] (60:\w * 3mm) circle (1pt);
        \foreach \w in {0,...,6}
            \draw[fill=black] (120:\w * 3mm) circle (1pt);
        \foreach \w in {0,...,6}
            \draw[fill=black] (150:\w * 3mm) circle (1pt);
        \foreach \w in {0,...,6}
            \draw[fill=black] (210:\w * 3mm) circle (1pt);
        \foreach \w in {0,...,6}
            \draw[fill=black] (240:\w * 3mm) circle (1pt);
        \foreach \w in {0,...,6}
            \draw[fill=black] (300:\w * 3mm) circle (1pt);
        \foreach \w in {0,...,6}
            \draw[fill=black] (330:\w * 3mm) circle (1pt);
    \end{tikzpicture}
    \caption{Radiale Linien, entstanden durch fouriertransformierte Projektionen (Vorlage:~\cite{kakslan}, S. 59)}
    \label{fig:fourier_scheibe_endlich}
\end{figure}

\subsection{Die gefilterte Rückprojektion}\label{sssec:filter_bp}

Das Fourier-Scheiben-Theorem stellt den Zusammenhang zwischen der fouriertransformierten Projektion und dem
fouriertransformierten Objekt entlang einer radialen Linie her. Daraus folgt, dass man mit genügend vielen
fouriertransformierten Projektionen, die unter unterschiedlichen Winkeln aufgenommen wurden, eine Annäherung an das
zweidimensional fouriertransformierte Objekt und damit (durch die inverse Fouriertransformation) an das Objekt selbst
erreichen kann. Dieses einfache Modell der Tomographie lässt sich jedoch nicht ohne Weiteres in dieser Form
implementieren (vgl.~\cite{kakslan}, S. 60).

Ein bekannter Ansatz zur Lösung dieses Problems ist die gefilterte Rückprojektion, die sehr genau und schnell zu
implementieren ist und sich aus dem Fourier-Scheiben-Theorem herleiten lässt (zur mathematischen Herleitung
siehe~\cite{kakslan}, S. 63 -- 68). Es ist dieser Schritt, der eine effiziente Implementierung der Rekonstruktion für
Computer ermöglicht; Kak und Slaney stellen fest: {\glqq}The derivation of this algorithm is perhaps one of the most
illustrative examples of how we can obtain a radically different computer implementation by simply rewriting the
fundamental expressions for the underlying theory{\grqq} (siehe~\cite{kakslan}, S. 60).

Die Grundlage der gefilterten Rückprojektion ist der Umstand, dass jede Projektion eine nahezu unabhängige Aufnahme
des Objekts darstellt. Die einzige Gemeinsamkeit der unter verschiedenen Winkeln aufgenommenen Projektionen ist die
Gleichheit im Punkt $F(0, 0)$. Transformiert man eine einzelne Projektion und nimmt an, dass es keine anderen
Projektionen gibt, so lässt sich durch die zweidimensionale inverse Fouriertransformation ein (verzerrtes) Objekt
rekonstruieren. Durch das Aufsummieren mehrerer transformierter Projektionen nimmt die Qualität des rekonstruierten
Objekts zu. Da die Fouriertransformation eine lineare Operation ist, lässt sich dieses Aufsummieren der Projektionen
auch im Ortsraum durchführen. In diesem Fall spricht man von der Rückprojektion (vgl.~\cite{kakslan}, S. 61).

Die Filterung lässt sich als Wichtung der Projektionen im Frequenzraum betrachten. Wie bereits in
Abschnitt~\ref{sssec:fourier_scheibe} erwähnt, nimmt die Anzahl der durch die (in endlicher Zahl vorliegenden) bekannten
Punkte im Frequenzraum mit steigender Distanz zum Ursprung ab. Durch die Wichtung der höheren Frequenzen der
Projektionen lassen sich die fehlenden Punkte approximieren, durch eine möglichst hohe Projektionszahl lässt sich der 
durch die fehlenden Informationen hervorgerufene Effekt weiter verringern (vgl.~\cite{kakslan}, S. 61).

Die gewichteten Projektionen lassen sich (nach der inversen Fouriertransformation) aufsummieren, um das Objekt zu
rekonstruieren. Zusammengefasst ergibt sich für die gefilterte Rückprojektion der folgende Algorithmus:

Für alle Winkel $\theta$:
\begin{enumerate}
    \item Projektion $P_{\theta}(t)$ aufnehmen
    \item $P_{\theta}(t)$ zu $S_{\theta}(w)$ fouriertransformieren
    \item $S_{\theta}(w)$ filtern
    \item $S_{\theta}(w)$ invers fouriertransformieren und zum bisherigen Ergebnis hinzu addieren
\end{enumerate}

Gegenüber der in Abschnitt~\ref{sssec:fourier_scheibe} erwähnten Interpolationsmethode bietet dieser Algorithmus den
Vorteil, dass direkt nach dem Aufnehmen der ersten Projektion begonnen werden kann. Daneben ist es einfacher, im
Ortsraum zu interpolieren, als wenn man dies im Frequenzraum täte, da im Ortsraum eine lineare Interpolation häufig
genügt (vgl.~\cite{kakslan}, S. 62).

\section{Der Feldkamp-Davis-Kress-Algorithmus}\label{ssec:fdk}

Der 1984 entwickelte \gls{fdk}~\cite{fdk} ist eine spezielle Ausprägung der gefilterten Rückprojektion für die
Computertomographie mit Kegelstrahlen. In diesem Abschnitt wird zunächst die zugrundeliegende Mathematik näher
erläutert, bevor die einzelnen Schritte des \gls{fdk} detaillierter betrachtet werden.

\subsection{Grundlagen}\label{sssec:fdk_math}

Die Idee des \gls{fdk} ist die Erweiterung der in Abschnitt~\ref{sssec:filter_bp} vorgestellten gefilterten
Rückprojektion auf die dritte Dimension. Ausgangspunkt ist eine Fächerstrahlprojektion (siehe
Abbildung~\ref{fig:fan_proj}), die jetzt im dreidimensionalen Raum betrachtet wird, wie in
Abbildung~\ref{fig:fdk_schema} dargestellt. Ausgehend von der Schnittgeraden der mittleren Ebene ($z = 0$) mit der
Detektorebene, also der Projektion, lassen sich die in der mittleren Ebene liegenden Punkte rekonstruieren. Verschiebt
man nun die Schnittgerade (den Fächer) bei gleichbleibender Quellposition derart, dass sie parallel zur Schnittgeraden
der mittleren Ebene mit dem Detektor bleibt (konstantes $z$, $z \neq 0$), so befinden sich die so erfassten Punkte
ebenfalls innerhalb einer Ebene. Diese neue Ebene lässt sich ebenfalls als eine mittlere Ebene betrachten, die
allerdings im Vergleich zur {\glqq}ursprünglichen{\grqq} mittleren Ebene gekippt ist.

Um die Rückprojektion einer Fächerstrahlprojektion auf die gekippte Ebene anwenden zu können, müssen die durch die
Verschiebung entstandenen Veränderungen berücksichtigt werden. Zum Aufnahmewinkel der Projektion kommt nun noch der
Winkel zwischen der gekippten Ebene und der mittleren Ebene hinzu, außerdem hat sich für den gekippten Strahl die
Distanz zwischen der Quelle und dem Detektor vergrößert. Berechnet man diese Faktoren mit ein, so lässt sich die
Rückprojektion von Fächerstrahlprojektionen auch im dreidimensionalen Raum durchführen (zur mathematischen Herleitung
siehe~\cite{fdk}, S. 614 -- 615). Aufgrund der Kegelform der von der Quelle ausgehenden Strahlen bezeichnet man sie auch
als \textit{Kegelstrahlen} (englisch \textit{cone-beam}; vgl.\ den Titel der Arbeit von Feldkamp, Davis und Kress,
\textit{Practical cone-beam algorithm}).

Es ist zu beachten, dass der \gls{fdk} keine exakte, sondern nur eine näherungsweise Lösung bietet: {\glqq}No rigorous
proof exists [\ldots], since the result is approximate{\grqq} (siehe~\cite{fdk}, S. 614).

\begin{figure}
    \centering
    \begin{tikzpicture}
        % Detektor
        \draw (3, 4, -4) -- (9, 4, 4) -- (9, -4, 4) -- (3, -4, -4) -- (3, 4, -4);

        % Rotationsachse unten
        \draw (0, -3, 0) -- (0, 0, 0);

        % mittlere Ebene
        \draw[fill=white] (-6, 0, -4) -- (0, 0, 4) -- (9, 0, 4) -- (3, 0, -4) -- (-6, 0, -4);

        % mittlere Achse
        \draw (-3, 0, 0) -- (6, 0, 0);
        \draw[fill=black] (-3, 0, 0) circle (1pt) node [below left] {$(-d, 0, 0)$};
        \draw[fill=black] (6, 0, 0) circle (1pt) node [above right] {$(d', 0, 0)$};
        
        % Volumen oben
        \draw[fill=white] (0, 1, -1) -- (-1, 1, -1) -- (0, 1, 1) -- (1, 1, 1) -- (0, 1, -1);
        \draw[fill=white] (0, 0, 1) -- (0, 1, 1) -- (1, 1, 1) -- (1, 0, 1);
        \draw[fill=white] (-1, 0, -1) -- (-1, 1, -1) -- (0, 1, 1) -- (0, 0, 1);

        % Volumen unten
        \draw[dashed] (0, 0, 1) -- (0, -1, 1) -- (1, -1, 1) -- (1, 0, 1);
        \draw[dashed] (-1, -1, -1) -- (-1, 0, -1);
        \draw[dashed] (-1, -1, -1) -- (0, -1, 1);
        \draw[dashed] (0, -1, -1) -- (1, -1, 1);

        % weitere Punkte
        \draw[fill=black] (0, 0, 0) circle (1pt);
        \draw[fill=black] (4, 1, -4) circle (1pt);
        \draw[fill=black] (-0.5, 5/14, -10/7) circle (1pt);

        % Quelle - Detektorpunkt
        \draw (-3, 0, 0) -- (4, 1, -4);

        % Beschriftungen
        \draw[->] (0, 0, 6) -- (0, 0, 3) node [pos=0, below left] {mittlere Ebene};
        \draw[->] (0, 3, -3) -- (4, 2, -4) node [pos=0, left] {Detektorebene};
        \draw[->] (2.5, -1, 0) -- (0, 0, 0) node [pos=0, right] {$(0, 0, 0)$};
        \draw[->] (2.5, 0.5, 0) -- (-0.5, 5/14, -10/7) node[pos=0, right] {$(x, y, z)$};

        % fehlende Linien
        \draw[dashed] (3, 0, -4) -- (3, -4, -4); % Detektor
        \draw[dashed] (0, -2, 0) -- (0, 0, 0); % Rotationsachse unten
        \draw[dashed] (0, 0, 0) -- (0, 1, 0); % Rotationsachse oben 1
        \draw (0, 1, 0) -- (0, 3, 0); % Rotationsachse oben 2
        \draw[dashed] (-1, 0, 0) -- (1, 0, 0); % mittlere Achse

    \end{tikzpicture}
    \caption{Schematische Darstellung eines 3D-Tomographiesystems (Vorlage:~\cite{fdk})}
    \label{fig:fdk_schema}
\end{figure}

\subsection{Geometrie}\label{sssec:fdk_geometrie}

Basierend auf den oben genannten Faktoren lässt sich ein dreidimensionales Tomographiesystem wie in
Abbildung~\ref{fig:fdk_geometrie} darstellen. Der Ausgangspunkt der Strahlung ist eine Quelle $S$ (\textit{source}), die
das Volumen $O$ (\textit{object}) unter einem Drehwinkel $\alpha_p$ mit einem \textit{kegelförmigen} Strahl
durchleuchtet und auf einem Detektor mit $N_h \cdot N_v$ \gls{pixel}n abbildet. Dabei stellt $d_{src}$ den Abstand
zwischen der Quelle und dem Rotationsmittelpunkt, also dem Zentrum des durchleuchteten Volumens, dar, während $d_{det}$
den Abstand zwischen dem Rotationsmittelpunkt und dem Detektor.

\begin{figure}[!htb]
\begin{tikzpicture}[
        scale=0.8,
        axis/.style={thick,->}
    ]
    % Quelle
    \coordinate (1) at (0, 0, 0);
    \filldraw[fill=black,draw=black] (1) circle (0.15cm) node[below left] {$S$};
    \draw[axis] (-2, 0.5, 0) -- (2, -0.5, 0) node[right] {$x_s$};
    \draw[axis] (2, 0.5, 0) -- (-2, -0.5, 0) node[left] {$y_s$};
    \draw[axis] (0, -4, 0) -- (0, 3, 0) node[above] {$z_s$};

    % Volumen
    \coordinate (2) at (8, 2, 0);
    \filldraw[fill=black,draw=black] (2) circle (0.15cm) node[below left] {$O$};
    \draw[axis] (5, 2, -3) -- (11, 2, 3) node[right] {$x_k$};
    \draw[axis] (10, 2, 0) -- (4, 2, 0) node[left] {$y_l$};
    \draw[axis] (8, -2, 0) -- (8, 4, 0) node[above] {$z_m$};

    \draw (8, 1, 1) -- (10, 1, 1);
    \draw (8, 1, 1) -- (8, 3, 1);
    \draw (8, 1, 1) -- (6, 1, -1);

    \draw (8, 3, 1) -- (10, 3, 1);
    \draw (8, 3, 1) -- (6, 3, -1);

    \draw (10, 1, 1) -- (10, 3, 1);

    \draw (6, 1, -1) -- (6, 3, -1);

    \draw (6, 3, -1) -- (8, 3, -1);

    \draw (10, 3, 1) -- (8, 3, -1);


    % Detektor
    \coordinate(3) at (14, 3.5, 0);
    \filldraw[fill=black,draw=black] (3) circle (0.15cm);
    \draw[axis] (10.25, 3.5, -2.5) -- (17.75, 3.5, 2.5) node[right] {$h$};
    \draw[axis] (14, -0.5, 0) -- (14, 7.5, 0) node[above] {$v$};
    \draw[fill=black!60!white,opacity=0.3] (10.75, 7, -2.16666) -- (17.25, 7, 2.16666) -- (17.25, 1, 2.16666)
                                           -- (10.75, 1, -2.16666) -- (10.75, 7, -2.16666);
    \draw[<->] (10.5, 1, -2.33333) -- (10.5, 7, -2.33333) node[pos=0.5, left] {$N_v$};
    \draw[<->] (10.75, 0.75, -2.16666) -- (17.25, 0.75, 2.16666) node[pos=0.5, below right] {$N_h$};

    % Abstände
    \draw (1) -- (3);
    \draw[<->] (0, -3.5, 0) -- (8, -1.5, 0) node[pos=0.5,below] {$d_{src}$};
    \draw[<->] (8, -1.5, 0) -- (14, 0, 0) node[pos=0.5,below] {$d_{det}$};

    % Kegelstrahlen
    \draw[dotted] (1) -- (16, 2, 2);
    \draw[dotted] (1) -- (16, 6, 2);
    \draw[dotted] (1) -- (12, 2, -2);
    \draw[dotted] (1) -- (12, 6, -2);

    % Kegelstrahlenabbild
    \draw[dotted] (16, 2, 2) -- (16, 6, 2) -- (12, 6, -2) -- (12, 2, -2) -- (16, 2, 2);

    % Winkel
    \draw (5, 2, 0) arc (180:195:2.8cm) node[pos=0.5, left] {$\alpha_p$};
\end{tikzpicture}
\caption{Geometrie der gefilterten Rückprojektion}
\label{fig:fdk_geometrie}
\end{figure}

\subsection{Wichtung}\label{sssec:fdk_wichtung}

Aufgrund der Kegelform der Strahlung hat jeder Strahl, der das Volumen durchleuchtet, beim Auftreffen auf den Detektor
einen im Vergleich zu den restlichen Strahlen unterschiedlich langen Weg zurückgelegt. Um die durch die
Streckenunterschiede bedingten Absorptionsveränderungen auszugleichen, ist es erforderlich, jeden Punkt der Projektion
(\gls{pixel}) zu wichten.
Dafür wird jedes \gls{pixel} mit den Koordinaten $(j, i)$ mit dem Wichtungsfaktor $w_{ij}$ multipliziert, der auf den
Abständen $d_{src}$ und $d_{det}$ sowie den vertikalen und horizontalen Distanzen des individuellen \gls{pixel}s vom
Ursprung des Detektorkoordinatensystems basiert:

\begin{equation}\label{eq:wichtung}
    w_{ij} = \frac{d_{det} - d_{src}}{\sqrt{(d_{det} - d_{src})^2 + h_j^2 + v_i^2}}
\end{equation}

\subsection{Filterung}\label{sssec:fdk_filter}

Aufgrund der in Abschnitt~\ref{filter_bp} genannten Gründe müssen die gewichteten Projektionen vor der Rückprojektion
gefiltert werden. Da es sich bei der Kegelstrahltomographie im Wesentlichen um eine erweiterte Fächerstrahltomographie
handelt, genügt es, die Projektionen zeilenweise zu filtern, wie man es bei einem {\glqq}Stapel{\grqq} von
Fächerstrahlprojektionen tun würde. Zu diesem Zweck müssen die Projektionen und der zum Einsatz kommende Filter
fouriertransformiert werden; zum Einsatz kommt dabei das Verfahren der schnellen Fouriertransformation (\textit{fast
Fourier transform}, FFT) nach Cooley und Tukey (vgl.~\cite{cooltuk}). Da dieses Verfahren nur mit einer Menge von
Elementen funktioniert, die einer Zweierpotenz entspricht, müssen die Projektionszeilen und der Filter auf die nächste
Zweierpotenz {\glqq}aufgerundet{\grqq} werden. Dazu wird, ausgehend von der Länge einer Projektionszeile $N_h$, die
Filterlänge $N_{hFFT}$ berechnet:

\begin{equation}
    N_{hFFT} = 2 \cdot 2^{\left\lceil \log_{2} N_h \right\rceil}
\end{equation}

Mit der so bestimmten Filterlänge lässt sich der Filter $r$ erzeugen:

\begin{equation}\label{eq:filter_gen}
    \begin{aligned}
        r(j) \text{ mit } j &\in \left[-\frac{N_{hFFT} - 2}{2}, \frac{N_{hFFT}}{2}\right]\\
        r(j) &=
            \begin{cases}
                \frac{1}{8} \cdot \frac{1}{\tau^2} & \quad \text{wenn } j = 0\\
                0 & \quad \text{wenn } j \text{ gerade}\\
                -\frac{1}{2j^2\pi^2\tau^2} & \quad \text{wenn } j \text{ ungerade}\\
            \end{cases}
    \end{aligned}
\end{equation}

Nun wird die zu filternde Zeile so lange mit $0$ aufgefüllt, bis die erweiterte Zeile $N_{hFFT}$ \gls{pixel} umfasst:

\begin{equation}
    \begin{aligned}
        p &: \text{ mit Nullen aufgefüllte Projektionszeile}\\
        p(0 \dots N_{h - 1}) &= \text{det}(0 \dots N_{h - 1})\\
        p(N_{h} \dots N_{hFFT}) &= 0
    \end{aligned}
\end{equation}

Im Anschluss werden sowohl der Filter $r$ als auch die erweiterte Projektionszeile $p$ in den komplexen Raum
transformiert und dort miteinander multipliziert:

\begin{equation}
    \begin{aligned}
        R &= \text{FFT}(r)\\
        P &= \text{FFT}(p)\\
        F &= P \cdot R \quad \text{sowohl für den reellen als auch den imaginären Teil}
    \end{aligned}
\end{equation}

Die so gefilterte Projektionszeile $F$ wird dann mit der inversen schnellen Fouriertransformation (IFFT) in den
reellen Raum zurücktransformiert und von den {\glqq}aufgefüllten{\grqq} Elementen bereinigt:

\begin{equation}
    \begin{aligned}
        f &= \text{IFFT}(F)\\
        \text{gefilterte Projektionszeile} &: f(0 \dots N_{h - 1})
    \end{aligned}
\end{equation}

\subsection{Rückprojektion}\label{sssec:backprojection}

Die gefilterten Projektionen können nun für die Rückprojektion verwendet werden. Dazu werden für jede vorhandene
Projektion $p$, die unter dem Drehwinkel $\alpha_p$ aufgenommen wurde, die folgenden Schritte ausgeführt:

\begin{itemize}
    \item berechne für jede \gls{voxel}koordinate $(x_k, y_l, z_m)$ deren rotierte Position $(s, t, z)$:
        \begin{equation}
            \begin{aligned}
                s &= x_k \cos \alpha_p + y_l \sin \alpha_p\\
                t &= -x_k \sin \alpha_p + y_l \cos \alpha_p\\
                z &= z_m
            \end{aligned}
        \end{equation}

    \item projiziere die rotierte \gls{voxel}koordinate $(s, t, z)$ auf den Detektor:
        \begin{equation}
            \begin{aligned}
                h' &= y' = t \cdot \frac{d_{det} - d_{src}}{s - d_{src}}\\
                v' &= z' = z \cdot \frac{d_{det} - d_{src}}{s - d_{src}}
            \end{aligned}
        \end{equation}

    \item interpoliere das Detektorsignal bei $(h', v')$:
        \begin{equation}
            \begin{aligned}
                det' = det(h', v')
            \end{aligned}
        \end{equation}

    \item führe die Rückprojektion für jedes \gls{voxel} $vol_{klm}$ aus:
        \begin{equation}
            \begin{aligned}
                vol_{klm} &= vol_{klm} + 0,5 \cdot det' \cdot u^2\\
                \text{mit } u &= \frac{d_{src}}{s - d_{src}}
            \end{aligned}
        \end{equation}
\end{itemize}

Nach dem Abschluss der Rückprojektion erhält man ein Volumen, dessen \gls{voxel} Aufschluss über seine innere Struktur
geben.

\chapter{Grundlagen der Programmierung mit CUDA{\textregistered}}\label{chap:grundlagen_cuda}

In diesem Kapitel wird die NVIDIA{\textregistered}-CUDA{\textregistered}-Plattform näher vorgestellt. Eingegangen wird
zunächst auf die historische Entwicklung, die die Einführung von CUDA{\textregistered} begünstigte bzw.\ erforderte. Im
Anschluss daran wird das CUDA{\textregistered}-Programmiermodell vorgestellt, das als technische Basis für die in
Kapitel~\ref{chap:umsetzung} beschriebene Implementierung dient.

\section{Programmierbare Grafikkarten}\label{sec:cu_prog_gpu}

Als NVIDIA{\textregistered} im Jahre 2006 seine \textit{Compute-Unified-Device-Architecture}-Plattform
(CUDA{\textregistered}) vorstellte, die die direkte Programmierung der NVIDIA{\textregistered}-Grafikkarten ermöglichte,
folgte die Firma damit einer Entwicklung, die in den ersten Jahren des neuen Jahrtausends begonnen hatte. Mit der
Einführung der NVIDIA{\textregistered} GeForce 3 im Jahre 2001 und der parallelen Veröffentlichung von \gls{directx} 8
bzw.\ \gls{opengl} \textit{Vertex-Shader}-Erweiterungen hatten Anwendungsentwickler erstmals Zugriff auf die
\textit{Shader}-Einheiten für die \textit{Vertex}- und \textit{Transform-\&-Lighting}-Berechnung. Spätere \gls{gpu}s,
die mit \gls{directx} 9 kompatibel waren, gestatteten eine noch flexiblere Programmierung, indem sie Entwicklern Zugriff
auf die \textit{Pixel-Shader} erlaubten und die Nutzung von Texturen im \textit{Vertex-Shader} zuließen. Die 2002
vorgestellte \gls{gpu} ATI Radeon 9700 verfügte über einen programmierbaren
24bit-Fließkommazahl-\textit{Pixel-Shader}-Prozessor, der mit \gls{directx} 9 und \gls{opengl} gesteuert werden
konnte, die GeForce{\textregistered} FX bot sogar 32bit-Fließkommazahl-Pixel-Prozessoren. Diese programmierbaren
Prozessoren waren Teil einer Entwicklung, die zu einer allmählichen Vereinheitlichung der auf der \gls{gpu} verbauten
Funktionseinheiten führte: während es auf den GeForce{\textregistered}-Serien 6800 und 7800 noch getrennte Prozessoren
für die \textit{Vertex}- und \textit{Pixel}-Berechnung gab, wurde in der 2005 erschienenen XBox 360 eine 
{\glqq}vereinheitlichte{\grqq} Grafikkarte verbaut, deren Prozessoreinheiten sowohl für die \textit{Vertex}- als auch
für die \textit{Pixel}-Berechnung geeignet waren (vgl.~\cite{kirkhwu}, S. 28 -- 29).

Durch diese Vereinheitlichung der \gls{gpu}-Prozessoren glich die Hardware mehr und mehr den aus dem \gls{hpc} bekannten
Parallelrechnern. Mit der Verfügbarkeit \gls{directx}-9-kompatibler \gls{gpu}s wurde diese Entwicklung zunehmend
auch in Forschungskreisen bekannt und man begann zu untersuchen, inwiefern sich die neue Hardware zur Lösung von
berechnungsintensiven Problemen aus den Bereichen der Natur- und Ingenieurswissenschaften einsetzen ließ. Die zu diesen
Zwecken nicht konstruierten \gls{gpu}s ließen sich jedoch nur über die vorhandenen Schnittstellen zur
Grafikprogrammierung ansteuern. Zur Nutzung der Hardwareressourcen musste ein Programmierer daher das zu lösende Problem
zunächst auf computergrafische Operationen abbilden, sodass die Berechnung dann mit \gls{opengl} oder \gls{directx}
durchgeführt werden konnte. Um beispielsweise eine mathematische Funktion mehrfach auszuführen, musste diese erst in
ein \textit{Pixel-Shader}-Programm umgeschrieben werden, während die zugehörigen Eingabedaten als Texturen
vorzuliegen hatten und die Ausgabedaten im der jeweiligen Grafikbibliothek eigenen Pixelformat zurückgegeben wurden.
Die Umschreibung in einen für die Verwendung von \textit{Pixel-Shadern} geeigneten Algorithmus hatte außerdem den
gravierenden Nachteil, dass Zugriffe auf beliebige Stellen im Speicher nicht möglich waren. Da ein \textit{Pixel-Shader}
als Ausgabedatum die Farbe eines \gls{pixel}s zurückliefert, besteht dazu aus Sicht der Grafikbibliothek auch keine
Notwendigkeit, da die Position des zugehörigen Pixels ja bereits bekannt ist (vgl.~\cite{kirkhwu}, S. 33).

Diese Beschränkungen erwiesen sich für generische numerische Berechnungen bald als zu restriktiv. Erschwerend kam hinzu, 
dass jeder Entwickler, der sich mit diesen technischen Limitierungen abfinden konnte, zusätzlich noch Wissen über die
Funktionsweise von \gls{opengl} und \gls{directx} benötigte, um das zu lösende Problem mit Hilfe von \gls{gpu}s
bewältigen zu lassen. Eine flächendeckende Akzeptanz von \gls{gpu}s als Beschleunigern war unter Forschern daher in den
ersten Jahren des \gls{gpgpu} nicht gegeben (vgl.~\cite{sandkand}, S. 6).

Die Situation änderte sich, als NVIDIA{\textregistered} 2006 die GeForce{\textregistered} 8800 GTX der Öffentlichkeit
präsentierte. Diese \gls{gpu} war nicht nur zum damals neuen \gls{directx} 10 kompatibel, sondern auch die erste
Grafikkarte, die mit \gls{cuda} ohne den Umweg über \gls{directx} oder \gls{opengl} direkt programmierbar war
(vgl.~\cite{sandkand}, S. 7). Programmierer hatten nun die Möglichkeit, \textit{datenparallele} Aspekte
(vgl. Abschnitt~\ref{sssec:cu_data_par}) des zu lösenden Problems zu deklarieren und von der \gls{gpu} ausführen zu
lassen. Die \textit{Shader}-Prozessoren hatten sich zu komplett programmierbaren Prozessoren entwickelt und verfügten
über einen Instruktionsspeicher, einen Instruktions-Cache und eine Instruktionskontrollogik. Diesen zusätzlichen
Hardwareoverhead konnte NVIDIA{\textregistered} dadurch reduzieren, dass mehrere Prozessoren sich den Instruktionscache
und die -kontrolllogik teilten. Diese Struktur funktioniert aufgrund des Hardware-Fokus auf die parallele Berechnung von
\gls{pixel}n für \gls{gpu}s gut. Zusätzlich war es nun möglich, auf beliebige Teile des Speichers zuzugreifen. Aus Sicht
des Entwicklers lag nun ein Programmiermodell vor, das eine Hierarchie paralleler Threads, Synchronisierungsmechanismen
und Barrieren sowie atomare Operationen bot und das Ganze in eine an C bzw.\ C++ angelehnte Sprache einbettete
(vgl.~\cite{kirkhwu}, S. 35).

\section{Das \gls{cuda}-Programmiermodell}

In diesem Abschnitt wird das der \gls{cuda}-Plattform zugrundeliegende Programmiermodell näher vorgestellt. Das
gedankliche Fundament dieses Modell ist die \textit{Datenparallelität}, die in Abschnitt~\ref{sssec:cu_data_par}
erläutert wird. Die konkreten Programmierkonzepte schließen sich in den danach folgenden Abschnitten an.

\subsection{Datenparallelität}\label{ssec:cu_data_par}

Moderne Programme bearbeiten häufig große Datenmengen und benötigen deswegen auf herkömmlichen Rechnern lange
Ausführungszeiten. Viele Anwendungen arbeiten dabei mit Daten, die Vorgänge der echten Welt ab- oder nachbilden, wie
beispielsweise einfache Bilder oder Filme oder die Bewegungen von Flüssigkeiten und Gasen unter bestimmten Umständen.
Fluglinien müssen ihre Flüge planen, wozu die Daten einer Vielzahl anderer Flüge, Besatzungen und Flughäfen herangezogen
werden, die voneinander unabhängig sind. Diese Unabhängigkeit der Daten ist die Voraussetzung für datenparallele
Algorithmen.

Als einfaches Beispiel für einen datenparallelen Algorithmus lässt sich die Addition zweier Vektoren $A$ und $B$
betrachten:

\begin{equation*}
    \left(
        \begin{array}{c}
            a_1\\
            a_2\\
            a_3
        \end{array}
    \right)
    +
    \left(
        \begin{array}{c}
            b_1\\
            b_2\\
            b_3
        \end{array}
    \right)
    =
    \left(
        \begin{array}{c}
            a_1 + b_1 \\
            a_2 + b_2 \\
            a_3 + b_3
        \end{array}
    \right)
    =
    \left(
        \begin{array}{c}
            c_1 \\
            c_2 \\
            c_3
        \end{array}
    \right)
\end{equation*}

In diesem Beispiel wird $c_1$ berechnet, indem $a_1$ und $b_1$ addiert werden, und $c_3$ ergibt sich aus der Addition
von $a_3$ und $b_3$. Die Additionen sind voneinander unabhängig und können parallel berechnet werden. Sehr große
Vektoren führen in diesem Beispiel daher zu einem hohen Maß an Datenparallelität.

Der Datenparallelität steht das Konzept der \textit{Task-Parallelität} gegenüber. Die Task-Parallelität kommt vor allem
in komplexeren Programmen zur Anwendung, in denen es mehrere unterschiedliche Aufgaben gibt, die parallel bearbeitet
werden können. Vorstellbar ist z.B.\ ein Programm zur Filterung von Bildern, in denen ein Task das die Bilder
nacheinander lädt, der zweite die Bilder nacheinander filtert und ein weiterer die gefilterten Bilder wieder
abspeichert (vgl.~\cite{kirkhwu}, S. 42 -- 43).

\subsection{Parallele Ausführung}

Die Ausführung eines \gls{cuda}-Programms beginnt auf dem \gls{host}. Wird ein \gls{kernel} aufgerufen, so wird auf dem
\gls{device} eine Anzahl an Threads gestartet. Die Summe aller auf dem \gls{device} gestarteten Threads bezeichnet
man als \textit{\gls{grid}}. Haben alle Threads eines \gls{kernel}s ihre Ausführung beendet, wird auch das \gls{grid}
beendet.
Das Starten eines \gls{kernel}s erzeugt typischerweise eine große Zahl von Threads. Im Gegensatz zur \gls{cpu}, auf der
das Erzeugen eines Threads in der Regel mehrere tausend Takte benötigt, ist der gleiche Vorgang auf der \gls{gpu} in
einigen wenigen Takten zu bewältigen (vgl.~\cite{kirkhwu}, S. 44 -- 45).

Das Beispiel der Vektoraddition aus Abschnitt~\ref{sssec:cu_data_par} lässt sich in C++ wie in
Quelltext~\ref{source:vec_add_cpp} dargestellt implementieren. 

\begin{code}
\begin{minted}[breaklines,breakafter=\,,fontsize=\small]{c++}
auto vec_add(const std::int32_t* A, const std::int32_t* B,
             std::int32_t* C, std::size_t size) -> void
{
    for(auto i = 0u; i < size; ++i)
        C[i] = A[i] + B[i];
}
\end{minted}
\captionof{listing}{Vektoraddition mit C++}
\label{source:vec_add_cpp}
\end{code}

In \gls{cuda} würde dagegen jeder Thread ein Element des Ausgabevektors $C$ berechnen, wie in
Quelltext~\ref{source:vec_add_cuda} gezeigt.

\begin{code}
\begin{minted}[breaklines,breakafter=\,,fontsize=\small]{cuda}
__global__ void vec_add(const std::int32_t* A, const std::int32_t* B,
                        std::int32_t* C, std::size_t size)
{
    auto i = blockIdx.x * blockDim.x + threadIdx.x;
    if(i < size)
        C[i] = A[i] + B[i];
}
\end{minted}
\captionof{listing}{Vektoraddition mit \gls{cuda}}
\label{source:vec_add_cuda}
\end{code}

Die Berechnung der Indices erfolgt in diesem Beispiel nicht über eine fortlaufend inkrementierte Variable, wie es in C++
der Fall wäre, sondern über die Hilfsvariablen \texttt{blockIdx}, \texttt{blockDim} und \texttt{threadIdx}. Wird ein
\gls{cuda}-\gls{kernel} ausgeführt, so entsteht auf dem \gls{device} ein Thread-\gls{grid}. Diese Threads werden in
einer zweistufigen Hierarchie angeordnet. Jedes \gls{grid} besteht aus mehreren Thread-Blöcken, die der Einfachheit
halber in der Folge als \textit{Blöcke} bezeichnet werden. Alle Blöcke innerhalb eines \gls{grid}s haben die gleiche
Größe, können jedoch maximal 1024 Threads umfassen; die Thread-Zahl der Blöcke wird auf dem \gls{host} vor der
Ausführung des \gls{kernel}s definiert. Für ein gegebenes \gls{grid} kann die Anzahl der Threads pro Block durch die 
Variable \texttt{blockDim} abgefragt werden, während der individuelle Block, zu dem ein bestimmter Thread gehört, durch
die Variable \texttt{blockIdx} identifiert werden kann. Innerhalb eines Blocks kann ein einzelner Thread über die
Variable \texttt{threadIdx} bestimmt werden. Durch die Kombination der drei Variablen, wie sie in
Quelltext~\ref{source:vec_add_cuda} dargestellt ist, lässt sich somit jeder Thread innerhalb des \gls{grid}s eindeutig
identifizieren. Da \gls{cuda} das Starten von zwei- und dreidimensionalen \gls{kernel}n unterstützt, kann über die
Felder \texttt{x}, \texttt{y} und \texttt{z} die Zahl der Blöcke bzw.\ Threads in der jeweiligen Dimension bestimmt
werden (vgl.~\cite{kirkhwu}, S. 54).

Ein weiterer Unterschied zur herkömmlichen C++-Programmierung ist das Schlüsselwort \texttt{\_\_global\_\_} vor der
Deklaration. Dieses Schlüsselwort dient zur Identifizierung einer Funktion als \gls{kernel}, also einer Funktion auf
dem \gls{device}, die vom \gls{host} aus aufgerufen werden kann. Daneben verwendet \gls{cuda} die Schlüsselworte
\texttt{\_\_device\_\_} und \texttt{\_\_host\_\_}, die anzeigen, auf welcher Hardware die jeweilige Funktion ausgeführt
werden kann bzw.\ soll. Eine \texttt{\_\_device\_\_}-Funktion kann dabei nur innerhalb eines \gls{kernel}s oder einer
weiteren \texttt{\_\_device\_\_}-Funktion aufgerufen werden, eine \texttt{\_\_host\_\_}-Funktion nur auf dem \gls{host}
-- sie entspricht also einer normalen C++-Funktion (vgl. Tabelle~\ref{table:cu_func_keywords} und~\cite{kirkhwu}, S.55).

\begin{table}
    \centering
    \begin{tabular}{| l | c | c |}
        \hline
        & Ausgeführt auf dem & Aufgerufen vom \\
        \hline
        \texttt{\_\_device\_\_ float device\_func()} & \gls{device} & \gls{device} \\
        \hline
        \texttt{\_\_global\_\_ void kernel\_func()} & \gls{device} & \gls{host} \\
        \hline
        \texttt{\_\_host\_\_ float host\_func()} & \gls{host} & \gls{host} \\
        \hline
    \end{tabular}
    \caption{\gls{cuda}-Schlüsselworte für die Funktionsdeklaration}
    \label{table:cu_func_keywords}
\end{table}

\subsection{Architektur}\label{sssec:cu_arch}

\subsection{Speicher}\label{sssec:cu_mem}

Wie auf einem klassischen Rechner mit seinem Hauptspeicher, diversen Caches und Prozessorregistern gibt es auch auf
einer \gls{cuda}-fähigen \gls{gpu} verschiedene Möglichkeiten zur Zwischenspeicherung von Daten.

Die verschiedenen Speichervarianten sind in Tabelle~\ref{table:cu_mem_hierarchie} zusammengefasst.

\begin{table}
    \centering
    \begin{tabular}{| l | c | c | c |}
        \hline
        Bezeichnung & Ort & Sichtbarkeit & Zugriff\\
        \hline
        \hline
        Register & On-Chip & pro Thread & lesen / schreiben \\
        \hline
        \texttt{\_\_shared\_\_} & On-Chip & pro Block & lesen / schreiben \\
        \hline
        \texttt{\_\_global\_\_} & Off-Chip & pro Kernel & lesen / schreiben \\
        \hline
        \texttt{\_\_constant\_\_} & Off-Chip & pro Kernel & nur lesen \\
        \hline
        \texttt{\_\_local\_\_} & Off-Chip & pro Thread & lesen / schreiben\\
        \hline
        Textur & Off-Chip & pro Kernel & nur lesen\\
        \hline
    \end{tabular}
    \caption{Die verschiedenen Varianten des \gls{cuda}-Speichers}
    \label{table:cu_mem_hierarchie}
\end{table}

\subsection{\glspl{stream}}\label{ssec:cu_streams}

\chapter{Umsetzung}

\section{Variantenvergleich}

\subsection{Bestehende Parallelisierungsstrategien und ihre Grenzen}

Von den in Abschnitt~\ref{ssec:par} genannten Ansätzen in der Literatur sind aufgrund ihrer Umsetzung für \gls{gpu}s die
Strategien von Xu et al.~\cite{xumuell}, Scherl et al.~\cite{scherlkeck} und Zhao et al.~\cite{zhao} von besonderem
Interesse für diese Arbeit.

Da Xu et al. 2004 mit ihrer Arbeit~\cite{xumuell} Neuland betraten, standen ihnen viele Methoden und Technologien, die
seitdem entwickelt wurden, noch nicht zur Verfügung. Die 2004 erschienenen \gls{gpu}s hatten im Vergleich zu heutigen
Grafikkarten sehr viel weniger Speicher; das damals beste verfügbare Produkt von NVIDIA{\textregistered}, die
GeForce{\textregistered} 6800 Ultra, konnte lediglich mit 512 MiB Speicher aufwarten und war nur über \gls{opengl} oder
\gls{directx} indirekt programmierbar~\cite{geforce6800}. Den begrenzten Speicher versuchte die Gruppe durch eine
Aufteilung des Volumens und eine schichtweise Rekonstruktion desselben unter Einbeziehung aller Projektionen möglichst
effizient zu nutzen. Aufgrund des technischen Fortschritts stehen uns heute andere Möglichkeiten zur Lösung dieses
Problems offen; so bietet etwa die NVIDIA{\textregistered} GeForce{\textregistered} GTX 1080 mit 8 GiB Speicher und der
Möglichkeit der direkten Programmierung mittels \gls{cuda} oder \gls{opencl} ganz andere Nutzungs- und
Berechnungsmöglichkeiten als ihre frühen Vorgänger~\cite{gtx1080}. Insbesondere ist es möglich, das ganze Volumen oder
Teile davon während der Berechnung im Speicher zu halten und dadurch häufige Kopien zwischen \gls{cpu}-Speicher und
\gls{gpu}-Speicher zu vermeiden.

Die Forschungsgruppe um Scherl~\cite{scherlkeck} baute auf der Idee, das Volumen im Speicher zu halten, auf und ging
stattdessen den Weg, jede Projektion einzeln in dieses Volumen zu projizieren. Zur Trennung bzw.\ Kapselung der
einzelnen Schritte entwickelten sie in einem vorherigen Schritt~\cite{scherlhopp} eine Pipeline-Struktur (nach Mattson
et al.~\cite{mattsan}). Jeder Schritt des \gls{fdk} wird dabei in einer eigenen Stufe (\textit{stage}) ausgeführt, die
in einem separaten Thread ausgeführt wird. Zur Kommunikation der Ergebnisse der einzelnen Stufen werden thread-sichere
Puffer verwendet, auf die die Eingabe- bzw.\ Ausgaberoutinen der Stufen zugreifen.

Die Grenzen bei dem vorgeschlagenen Verfahren der Gruppe um Zhao et al.~\cite{zhao} sind vor allem praktischer Natur.
Das von ihnen vorgestellte Modell sieht vor, Symmetrien auszunutzen und dadurch Rechenzeit einzusparen. Sie machen sich
dabei den Umstand zunutze, dass die auf den Detektor projizierte Koordinate eines \gls{voxel}s der um 90° rotierten
Projektionskoordinate des um den gleichen Betrag rotierten Voxels entspricht. Auf diese Weise lassen sich durch eine
Berechnung die Detektorkoordinaten von vier \gls{voxel}n finden, was eine Verkürzung der Rechenzeit verspricht. In der
Praxis scheitert dieses Verfahren an dem mechanischen Aufbau üblicher CT-Scanner. Da entweder Quelle und Detektor oder
aber das Untersuchungsobjekt rotiert werden müssen, kommt es durch Fehler in der Mechanik häufig dazu, dass Aufnahmen
doppelt gemacht oder übersprungen werden; auch kann es passieren, dass die Winkelschritte zwischen zwei Aufnahmen nicht
immer einheitlich sind.

\subsection{Das Problem des GPU-Speichers}

\subsection{Heterogene GPU-Systeme und effiziente Arbeitsteilung}

\section{Implementierung und Optimierung}

In diesem Abschnitt werden die Implementierung und die Optimierung des \gls{fdk} beschrieben. Zunächst werden die
Implementierungs- und Optimierungsziele sowie die Einflüsse der realen Welt auf das Modell vorgestellt und im Anschluss
daran die Umsetzung der Stufen \textit{Wichtung} und \textit{Filterung} gezeigt. Der Abschnitt schließt mit der
Implementierung der Rückprojektion.

\subsection{Implementierungs- und Optimierungsziele}

\begin{itemize}
    \item Wartbarkeit
    \item Portabilität
    \item sinnvolle Geschwindigkeit
    \item nutzbar auf Laptop / Workstation / GPU-Cluster
\end{itemize}

\subsection{Einflüsse der realen Welt}

Die in Kapitel~\ref{chap:grundlagen} gezeigten Überlegungen zur gefilterten Rückprojektion und dem \gls{fdk} sind in
dieser Form rein theoretischer Natur. Die Anwendung dieser Modelle auf die reale Welt ist mit einigen Schwierigkeiten
bzw.\ Einflüssen verbunden, die im Folgenden näher vorgestellt werden sollen.

\subsubsection{Detektorgeometrie}

Der Detektor übt aufgrund der durch ihn gewonnenen Informationen (in Form der Projektionen) einen großen Einfluss auf
die gefilterte Rückprojektion aus. Seinem Aufbau muss daher bei der Implementierung des \gls{fdk} besondere
Aufmerksamkeit zukommen.

Der Detektor hat eine feste Breite und Höhe (gemessen in Millimetern) und besteht aus einer zweidimensionalen Anordnung
von \gls{pixel}n (siehe Abbildung~\ref{fig:det_geometrie}). In der horizontalen Richtung verfügt er über $N_h$
\gls{pixel}, in der vertikalen Richtung sind es $N_v$ \gls{pixel}.

Jedes \gls{pixel} hat eine physische Breite $d_h$ und Höhe $d_v$ (gemessen in Millimetern); äquivalent lassen sich diese
Ausdehnungen als horizontale (vertikale) Abstände zwischen den \gls{pixel}zentren betrachten (siehe
Abbildung~\ref{fig:det_pixel}).

Spannt man nun über dem Detektor ein Koordinatensystem auf (mit dessen Zentrum als Ursprung, siehe
Abbildung~\ref{fig:det_koord}), wobei $h_{min}$, $h_{max}$, $v_{min}$ und $v_{max}$ den horizontalen (vertikalen)
Abstand von den Detektorrändern bis zur Detektormitte in Millimetern angeben, so ergeben sich die folgenden
Zusammenhänge:

\begin{equation}\label{eq:det_h}
    \begin{aligned}
        h_{max} - h_{min} &= N_h \cdot d_h\\
        h_{min} + \frac{N_h \cdot d_h}{2} &= 0
    \end{aligned}
\end{equation}

\begin{equation}\label{eq:det_v}
    \begin{aligned}
        v_{max} - v_{min} &= N_v \cdot d_v\\
        v_{min} + \frac{N_v \cdot d_v}{2} &= 0
    \end{aligned}
\end{equation}

\begin{figure}[!tb]
    \centering
    \begin{minipage}[b]{.5\textwidth}
        \centering
        \begin{tikzpicture}
            \draw[fill=black!60!white,opacity=0.3] (-3, 3) -- (3, 3) -- (3, -3) -- (-3, -3) -- (-3, 3);
            \draw[fill=black] (0,0) circle (0.5pt);

            % Lupe
            \draw (-2.4, 2.4) circle (1.5cm);

            % Pixellinien vertikal
            \draw (-2.75, 3) -- (-2.75, 0.95);
            \draw (-2.5, 3) -- (-2.5, 0.9);
            \draw (-2.25, 3) -- (-2.25, 0.9);
            \draw (-2, 3) -- (-2, 0.95);
            \draw (-1.75, 3) -- (-1.75, 1.05);
            \draw (-1.5, 3) -- (-1.5, 1.2);
            \draw (-1.25, 3) -- (-1.25, 1.43);
            \draw (-1, 2.94) -- (-1, 1.85);

            % Pixellinien horizontal
            \draw (-3, 2.75) -- (-0.95, 2.75);
            \draw (-3, 2.5) -- (-0.9, 2.5);
            \draw (-3, 2.25) -- (-0.9, 2.25);
            \draw (-3, 2) -- (-0.95, 2);
            \draw (-3, 1.75) -- (-1.05, 1.75);
            \draw (-3, 1.5) -- (-1.2, 1.5);
            \draw (-3, 1.25) -- (-1.43, 1.25);
            \draw (-3, 1) -- (-1.85, 1);

            % Pfeile und Beschriftungen
            \draw[<->] (-3, -3.5) -- (3, -3.5) node[pos=0.5, below] {Breite [mm] / $N_h$ [px]};
            \draw[<->] (3.5, 3) -- (3.5, -3) node[pos=0.5, sloped, above] {Höhe [mm] / $N_v$ [px]};
        \end{tikzpicture}
        \captionof{figure}{Detektorgeometrie}
        \label{fig:det_geometrie}
    \end{minipage}%
    \begin{minipage}[b]{.5\textwidth}
        \centering
        \begin{tikzpicture}
            \fill[black!60!white,opacity=0.3] (-2, 2) rectangle (3, -3);
            \draw (-2, 2) -- (2, 2) -- (2, -2) -- (-2, -2) -- (-2, 2);
            \draw (2, 2) -- (3, 2);
            \draw (-2, 0) -- (3, 0);
            \draw (0, 2) -- (0, -3);
            \draw (-2, 2) -- (-2, -3);
            \draw (2, -2) -- (2, -3);
            \draw (2, -2) -- (3, -2);
            \draw[fill=black] (-1, 1) circle (0.5pt);
            \draw[fill=black] (1, 1) circle (0.5pt);
            \draw[fill=black] (1, -1) circle (0.5pt);
            \draw[fill=black] (-1, -1) circle (0.5pt);

            % Verlängerungen
            \draw[dotted] (-1, 1) -- (-3, 1);
            \draw[dotted] (-1, -1) -- (-3, -1);
            \draw[dotted] (-1, -1) -- (-1, -4);
            \draw[dotted] (1, -1) -- (1, -4);

            % Pfeile und Beschriftungen
            \draw[->] (2, 2.5) -- (1, 2) node[pos=0, above] {Detektorrand};
            \draw[->] (-2, 2.5) -- (-1, 1) node[pos=0, above] {Pixelzentrum};
            \draw[<->] (-2.5, 1) -- (-2.5, -1) node[pos=0.5, left] {$d_v$};
            \draw[<->] (-3.25, 2) -- (-3.25, 0) node[pos=0.5, left] {$d_v$};
            \draw[<->] (-1, -3.5) -- (1, -3.5) node[pos=0.5, below] {$d_h$};
            \draw[<->] (-2, -4.25) -- (0, -4.25) node[pos=0.5, below] {$d_h$};
        \end{tikzpicture}
        \captionof{figure}{Pixelgeometrie}
        \label{fig:det_pixel}
    \end{minipage}
\end{figure}

\begin{figure}[!tb]
    \centering
    \begin{minipage}[b]{.5\textwidth}
        \centering
        \begin{tikzpicture}[axis/.style={thick,->}]
            \draw[fill=black!60!white,opacity=0.3] (-3, 3) -- (3, 3) -- (3, -3) -- (-3, -3) -- (-3, 3);
            \draw[fill=black] (0,0) circle (0.5pt);

            % Achsen
            \draw[axis] (0, -3) -- (0, 3) node [pos=0.9, right] {$v$};
            \draw[axis] (-3, 0) -- (3, 0) node [pos=0.9, below] {$h$};

            % Markierungen
            \draw[thick] (-3, 0.5) -- (-3, -0.5);
            \draw[thick] (3, 0.5) -- (3, -0.5);
            \draw[thick] (-0.5, 3) -- (0.5, 3);
            \draw[thick] (-0.5, -3) -- (0.5, -3);

            % Beschriftungen
            \draw[->] (1.5, 1.5) -- (3, 0) node [pos=0, left] {$h_{max}$};
            \draw[->] (-1.5, -1.5) -- (-3, 0) node [pos=0, right] {$h_{min}$};
            \draw[->] (-1.5, 1.5) -- (0, 3) node [pos=0, left] {$v_{max}$};
            \draw[->] (1.5, -1.5) -- (0, -3) node [pos=0, right] {$v_{min}$};
        \end{tikzpicture}
        \caption{Detektorkoordinatensystem}
        \label{fig:det_koord}
    \end{minipage}%
    \begin{minipage}[b]{.5\textwidth}
        \centering
        \begin{tikzpicture}
            \draw[fill=black!60!white,opacity=0.3] (-3, 3) -- (3, 3) -- (3, -3) -- (-3, -3) -- (-3, 3);
            \draw[fill=black] (0,0) circle (0.5pt);

            % Verschiebung
            \draw[thick] (-1.75, 1.5) -- (-1.25, 1.5);
            \draw[thick] (-1.5, 1.75) -- (-1.5, 1.25);

            % Pfeile & Beschriftung
            \draw[<->] (0, 1.5) -- (0, 0) node[pos=0.5,right] {$\Delta v$};
            \draw[<->] (0, 0) -- (-1.5, 0) node[pos=0.5,above] {$\Delta h$};
        \end{tikzpicture}
        \caption{Verschiebungsgeometrie}
        \label{fig:off_geometrie}
    \end{minipage}
\end{figure}

\subsubsection{Verschiebungen}

In einem idealen Modell sind die Strahlungsquelle und der Detektor genau aufeinander ausgerichtet, das heißt, dass der
Mittelpunkt der Strahlungsquelle und der Mittelpunkt des Detektors auf derselben Achse liegen. Durch den mechanischen
Aufbau einer realen Computertomographie-Anlage und deren händischer Justierung kommt es allerdings zu sowohl
einer horizontalen Verschiebung $\Delta h$ als auch einer vertikalen Verschiebung $\Delta v$ dieser Achse (siehe
Abbildung~\ref{fig:off_geometrie}). Von der Strahlungsquelle ausgehend trifft sie somit nicht mehr auf das Zentrum des
Detektors, sondern auf einen anderen Teil. Nimmt man Bezug auf die Detektorgeometrie, so müssen diese Verschiebungen
entsprechend berücksichtigt werden, da ansonsten ein verfälschtes Ergebnis berechnet wird. Dazu müssen die
Formeln~\ref{eq:det_h} und~\ref{eq:det_v} wie folgt umgeschrieben werden:

\begin{equation} 
    h_{min} + \frac{N_h \cdot d_h}{2} + \Delta h = 0
\end{equation}

\begin{equation}
    v_{min} + \frac{N_v \cdot d_v}{2}  + \Delta v = 0
\end{equation}

\subsubsection{Fehlende Projektionen}

Es leuchtet ein, dass Projektionen, die im Abstand von 180° aufgenommen wurden, das durchleuchtete Objekt
spiegelverkehrt darstellen. In der Theorie würde es also ausreichen, einen Halbkreis um das Objekt abzufahren, um alle
erforderlichen Informationen für die Rückprojektion zu gewinnen. In der Praxis kann es aufgrund mechanischer Fehler bei
der Rotation des Quelle-Detektor-Aufbaus allerdings dazu kommen, dass einzelne Projektionen übersprungen werden oder die
Winkelabstände zwischen zwei Projektionen verschieden groß sind. Das Abfahren eines Vollkreises dient dazu, die so
entstandenen Fehler durch Redundanzen zu minimieren.

\subsection{Geometrische Berechnungen}

\begin{itemize}
    \item Berechnung der Volumengeometrie
    \item Aufteilung in Teilvolumen
\end{itemize}

\subsection{Implementierung der Vorstufen}

\subsubsection{Wichtung}

Die Grundlage der Wichtungsoperation ist die in Abschnitt~\ref{sssec:fdk_wichtung} vorgestellte
Formel~\ref{eq:wichtung}:

\begin{equation*}
    w_{ij} = \frac{d_{det} - d_{src}}{\sqrt{(d_{det} - d_{src})^2 + h_j^2 + v_i^2}}
\end{equation*}

Es ist leicht zu sehen, dass sich der Wichtungsfaktor $w_{ij}$ zwar pro \gls{pixel} ändert, aber nicht von der konkreten
Projektion abhängig ist. Es ist daher möglich, die Berechnung der Wichtungsfaktoren am Anfang des Programms genau einmal
durchzuführen und in einer Wichtungsmatrix \texttt{m} zu speichern (siehe Quelltext~\ref{source:impl_gen_mat}). Die
Berechnung der Wichtungsmatrix hängt von mehreren geometrischen Parametern ab (vgl.\
Abschnitt~\ref{sssec:fdk_geometrie} und Abbildungen~\ref{fig:det_geometrie},~\ref{fig:det_pixel},~\ref{fig:det_koord},~\ref{fig:off_geometrie}):

\begin{itemize}
    \item \texttt{dim\_x}: Anzahl der \gls{pixel} in horizontaler Richtung. Entspricht der Anzahl der Detektorpixel in
          horizontaler Richtung $N_h$
    \item \texttt{dim\_y}: Anzahl der \gls{pixel} in vertikaler Richtung. Entspricht der Anzahl der Detektorpixel in
          vertikaler Richtung $N_v$
    \item \texttt{h\_min}: horizontaler Abstand vom Detektorrand zum Detektorzentrum in mm.
    \item \texttt{v\_min}: vertikaler Abstand vom Detektorrand zum Detektorzentrum in mm.
    \item \texttt{d\_sd}: Abstand von der Quelle zum Detektor. Entsprich der Differenz der Strecken $d_{det}$ (Abstand
          zwischen dem Objekt und dem Detektor) und $d_{src}$ (Abstand zwischen der Quelle und dem Objekt) bzw.\ der
          Summe ihrer Beträge:
          
          \begin{equation*}
              d_{det} - d_{src} = |d_{det}| + |d_{src}|
          \end{equation*}

    \item \texttt{l\_px\_row}: horizontale Länge eines \gls{pixel}s, also der horizontale Abstand zwischen den
          Mittelpunkten zweier aufeinanderfolgender \gls{pixel}. Entspricht der horizontalen Länge eines Detektorpixels
          $d_h$.
    \item \texttt{l\_px\_col}: vertikale Länge eines \gls{pixel}s, also der vertikale Abstand zwischen den
          Mittelpunkten zweier aufeinanderfolgender \gls{pixel}. Entspricht der vertikalen Länge eines Detektorpixels
          $d_v$.
\end{itemize}

\begin{code}
\begin{minted}[breaklines,breakafter=\,,fontsize=\small]{cuda}
__global__ void matrix_generation_kernel(float* m,
    std::uint32_t dim_x, std::uint32_t dim_y, std::size_t pitch,
    float h_min, float v_min, float d_sd, float l_px_row,
    float l_px_col)
{
    auto s = blockIdx.x * blockDim.x + threadIdx.x;
    auto t = blockIdx.y * blockDim.y + threadIdx.y;

    if((s < dim_x) && (t < dim_y))
    {
        auto row = reinterpret_cast<float*>(
            reinterpret_cast<char*>(m) + t * pitch);

        // Detektorkoordinaten in mm
        const auto h_s = (l_px_row / 2.f) + s * l_px_row + h_min;
        const auto v_t = (l_px_col / 2.f) + t * l_px_col + v_min;

        // berechne Wichtungsfaktor
        row[s] = d_sd * rsqrtf(d_sd * d_sd + h_s * h_s + v_t * v_t);
    }
}
\end{minted}
\captionof{listing}{Generierung der Wichtungsmatrix}
\label{source:impl_gen_mat}
\end{code}

Bei der Wichtung einer Projektion \texttt{p} kann der jeweilige Wichtungsfaktor aus der generierten Matrix \texttt{m}
ausgelesen und auf das zugehörige \gls{pixel} angewendet werden (siehe Quelltext~\ref{source:impl_weighting}). Die so
gewichtete Projektion wird dann im folgenden Schritt gefiltert.

\begin{code}
\begin{minted}[breaklines,breakafter=\,,escapeinside=||,fontsize=\small]{cuda}
__global__ void weighting_kernel(float* p, const float* m,
    std::uint32_t dim_x, std::uint32_t dim_y, std::size_t pitch,
    std::size_t m_pitch)
{
    auto s = blockIdx.x * blockDim.x + threadIdx.x;
    auto t = blockIdx.y * blockDim.y + threadIdx.y;

    if((s < dim_x) && (t < dim_y))
    {
        auto p_row = |\textbf{\textcolor{keyword-green}{reinterpret\_cast}}|<float*>(
            |\textbf{\textcolor{keyword-green}{reinterpret\_cast}}|<char*>(p) + t * pitch);
        auto m_row = |\textbf{\textcolor{keyword-green}{reinterpret\_cast}}|<const float*>(
            |\textbf{\textcolor{keyword-green}{reinterpret\_cast}}|<const char*>(m) + t * m_pitch);

        // Wichtung
        p_row[s] *= m_row[s];
    }
}
\end{minted}
\captionof{listing}{Wichtung einer Projektion}
\label{source:impl_weighting}
\end{code}

\subsubsection{Filterung}

Dem in Abschnitt~\ref{sssec:fdk_filter} vorgestellten Algorithmus entsprechend, folgt die Implementierung des
Filterschrittes dem nachstehenden Schema:

\begin{enumerate}
    \item einmalige Erzeugung und Fouriertransformation des Filters
    \item zeilenweise Fouriertransformation der Projektion
    \item Anwendung des Filters auf die jeweilige Projektionszeile im komplexen Raum
    \item inverse zeilenweise Fouriertransformation der Projektion
\end{enumerate}

Die Implementierung der Filtergenerierung entspricht der Formel~\ref{eq:filter_gen} und kann dem im Anhang befindlichen
Quelltext~\ref{app:filter_gen} entnommen werden. Dieser Filter wird dann zeilenweise auf jede Projektion angewendet.
Dazu werden der Filter und die einzelnen Projektionszeilen mit der \gls{cufft}-Bibliothek zunächst fouriertransformiert.
Im komplexen Raum werden dann die einzelnen Elemente der transformierten Projektionszeile mit den korrespondierenden
Elementen des transformierten Filters multipliziert (siehe Quelltext~\ref{source:impl_filter}). Ist dieser Vorgang
abgeschlossen, wird die Projektion wieder zurücktransformiert und normalisiert (siehe den angehängten
Quelltext~\ref{app:filter_norm}). Die Projektion ist dann bereit für die Rückprojektion.

\begin{code}
\begin{minted}[breaklines,breakafter=\,,fontsize=\small]{cuda}
__global__ void filter_application_kernel(
    cufftComplex* __restrict__ data,
    const cufftComplex* __restrict__ filter,
    std::uint32_t filter_size, std::uint32_t data_height,
    std::size_t pitch)
{
    auto x = blockIdx.x * blockDim.x + threadIdx.x;
    auto y = blockIdx.y * blockDim.y + threadIdx.y;

    if((x < filter_size) && (y < data_height))
    {
        auto row = reinterpret_cast<cufftComplex*>(
            reinterpret_cast<char*>(data) + y * pitch);

        row[x].x *= filter[x].x;
        row[x].y *= filter[x].y;
    }
}
\end{minted}
\captionof{listing}{Filterung einer Projektion}
\label{source:impl_filter}
\end{code}

\subsection{Implementierung der gefilterten Rückprojektion}

\begin{itemize}
    \item welche Konstanten und Variablen gibt es
    \item welche Schwierigkeiten können auftreten
\end{itemize}

\chapter{Ergebnisse und Diskussion}

\section{Leistungsmessung}

\subsection{Gesamtprogramm}

Messungen des Gesamtprogramms (Datentransfers, alle Stufen)

\subsection{Teilstufen}

grobe Messungen der Teilstufen (Wichtung, Filterung, Rückprojektion)

Ergebnis: Rückprojektion braucht am längsten

\subsection{Eigenschaften des Rückprojektionskernels}

In Abschnitt~\ref{ssec:opti_ueber} wurde als Implementierungsziel unter anderem eine möglichst hohe \gls{gpu}-Auslastung
gefordert. Um diese Auslastung zu erreichen, ist ein möglichst geringer Verbrauch der zwischen den Threads geteilten
Ressourcen erforderlich. Da der \textit{Shared Memory} in der Implementierung nicht verwendet wurde, zielte die
Umsetzung dieser Forderung auf einen möglichst geringen Registerverbrauch ab.

Wie in Abbildung~\ref{fig:kernel_props} zu sehen ist, benötigt der Rückprojektions-\gls{kernel} 19 Register
(grün markiert). Auf der hier verwendeten GeForce GTX 1080 ist somit eine theoretische Auslastung der gesamten \gls{gpu}
möglich, praktisch wird die \gls{gpu} durch die Rückprojektion zu 88,5\% ausgelastet (rot markiert).

Ein weiteres der drei Optimierungsziele, die in Abschnitt~\ref{ssec:opti_ueber} genannt wurden, ist die Verkürzung der
Wartezeit zwischen dem Ende einer Rückprojektion und dem Beginn der nachfolgenden Rückprojektion. Zu diesem Zweck wird
die Rückprojektion in einem eigenen Thread und \gls{cuda}-Stream ausgeführt, sodass die vorherigen Operationen (Laden,
Wichten, Filtern) parallel zur Rückprojektion ausgeführt werden können. Der Rückprojektionsthread entnimmt die nächste
Projektion vom Stapel der noch ausstehenden Projektionen, wandelt diese in eine \gls{cuda}-Textur um und startet dann
den Rückprojektions-\gls{kernel}. Wie Abbildung~\ref{fig:kernel_wait} zeigt, ist es gelungen, die Zeit zwischen zwei
Rückprojektionen, die durch die langen, grün-blauen Balken dargestellt werden, auf 2 Millisekunden zu beschränken. Bei
einem üblichen Datensatz von 1440 Projektionen ergibt sich also eine Gesamtwartezeit von 2,88 Sekunden.

Abbildung~\ref{fig:kernel_wait} zeigt allerdings auch, dass diese Wartezeit noch nicht ideal ist. Das Ziel, die Wichtung
und die Filterung, die durch die kleineren, bunten Balken repräsentiert werden, parallel zur Rückprojektion
durchzuführen, konnte nicht erreicht werden. Der Grund dafür ist vermutlich die Anzahl der Threads, die vom
Rückprojektions-\gls{kernel} gestartet werden. Bei einem Volumen mit 1070 x 1070 x 1033 \gls{voxel}n, denen jeweils ein
Thread zugeordnet wird, werden 1.182.681.700 Threads benötigt. Die GTX 1080 verfügt über 20 \gls{sm}s, von denen jeder
bis zu 2.048 Threads gleichzeitig ausführen kann und kann somit bis zu 40.960 Threads gleichzeitig verwalten. Die
\gls{gpu} ist daher gezwungen, der Rückprojektion alle verfügbaren Threads zuzuweisen, was zur Folge hat, dass keine
weiteren \gls{kernel} ausgeführt werden können. Wenn auch die parallele Ausführung der Rückprojektion und der vorherigen
Schritte nicht gelungen ist, ist auf der Abbildung jedoch zu sehen, dass die Ausführung der \gls{kernel} mit geringerem
Ressourcenbedarf sich stellenweise überlappt.

Die Wartezeit zwischen zwei Rückprojektionen, die über die gesamte Laufzeit akkumuliert ca. 3 Sekunden einnimmt, hat
also einen akzeptablen Wert erreicht, der dem alltäglichen Gebrauch nicht im Wege steht. Auch wenn an dieser Stelle noch
Optimierungspotential vorhanden ist, wäre selbst eine vollständige Eliminierung der Wartezeit vernachlässigbar.

\begin{figure}
    \includegraphics[width=\linewidth]{img/kernel_properties}
    \caption{Eigenschaften des Rückprojektionskernels}
    \label{fig:kernel_props}
\end{figure}

\begin{figure}
    \includegraphics[width=\linewidth]{img/timeline_compute3}
    \caption{Zeit zwischen zwei Rückprojektionen}
    \label{fig:kernel_wait}
\end{figure}

\subsection{Laufzeitverhalten}

Die parallele Berechnung verschiedener Teile des Volumens durch den Einsatz mehrerer \gls{gpu}s war ebenfalls ein
Implementierungsziel. Die besondere Herausforderung bestand darin, sowohl mehrere \gls{gpu}s des gleichen Typs als auch
unterschiedliche \gls{gpu}s zu unterstützen.

Wie Abbildung~\ref{fig:kernel_multi_compute} zeigt, ist es gelungen, die Ausführung parallel auf mehreren \gls{gpu}s
durchzuführen. Der gewählte Ansatz zur Lastverteilung (siehe Abschnitt~\ref{}) führt jedoch bei unterschiedlich
leistungsstarken \gls{gpu}s dazu, dass die stärkere \gls{gpu} vor der schwächeren fertig ist und dann auf diese warten
muss, wie in Abbildung~\ref{fig:kernel_multi_bad} zu sehen ist.

In Abbildung~\ref{fig:laufzeit_gpus} ist das Laufzeitverhalten unterschiedlicher \gls{gpu}s dargestellt. Es ist klar
zu sehen, dass der gemeinsame Einsatz der Tesla K20c und der GTX 1080 in diesem Fall sogar zu einer längeren Laufzeit
führt, als der alleinige Einsatz der GTX 1080. Die statische Lastverteilung, wie sie in dieser Arbeit beschrieben ist,
sollte daher zukünftig auf heterogenen \gls{gpu}-Systemen durch andere Methoden abgelöst werden. Denkbar ist
beispielsweise der Einsatz von \textit{Machine-Learning}-Techniken, durch die die optimale Lastverteilung iterativ
ermittelt wird.

\begin{figure}
    \includegraphics[width=\linewidth]{img/timeline_multi_compute}
    \caption{Parallele Ausführung auf zwei \gls{gpu}s}
    \label{fig:kernel_multi_compute}
\end{figure}

\begin{figure}
    \includegraphics[width=\linewidth]{img/timeline_multi_bad}
    \caption{Der gewählte, statische Lastverteilungsansatz führt zu Wartezeiten}
    \label{fig:kernel_multi_bad}
\end{figure}

\begin{figure}
    \centering
    \begin{tikzpicture}
        \begin{axis}[width=\textwidth,
                     xlabel={Volumengröße [Voxel]},
                     symbolic x coords={133 x 133 x 129,267 x 267 x 258,535 x 535 x 516, 1070 x 1070 x 1033},
                     xtick=data,
                     ylabel={Laufzeit [s]},
                     legend pos=north west]

             \addplot table[x=Volumengroesse,y=GTX1080,col sep=comma] {data/mehreregpus.csv};
             \addplot table[x=Volumengroesse,y=K20c,col sep=comma] {data/mehreregpus.csv};
             \addplot table[x=Volumengroesse,y=GTX1080K20c,col sep=comma] {data/mehreregpus.csv};
             \legend{GTX 1080,Tesla K20c,GTX \& Tesla};
        \end{axis}
    \end{tikzpicture}
    \caption{Laufzeit mit mehreren \gls{gpu}s}
    \label{fig:laufzeit_gpus}
\end{figure}

\section{Vergleich mit der Literatur}

Der Vergleich mit den oben vorgestellten Ansätzen der Literatur ist ebenfalls von Interesse. Da Zhao et al.\ in ihrer
Arbeit nur Zeiten für die Rückprojektion angeben, werden im Vergleich mit der von ihnen vorgestellten Variante ebenfalls
nur die Zeiten der Rückprojektion berücksichtigt. Scherl et al.\ beziehen dagegen auch die Datenein- und -ausgabe mit
ein. Zur Messung wurde von Zhao et al.\ die 2007 erschienene NVIDIA Quadro FX4600 eingesetzt, die über 768 MiB Speicher
verfügt; Scherl et al.\ verwendeten die 2006 vorgestellte NVIDIA GeForce 8800 GTX, die ebenfalls mit 768 MiB Speicher
ausgerüstet ist. Für den Vergleich fand in dieser Arbeit die 2016 präsentierte NVIDIA GeForce GTX 1080 Verwendung, die
mit 8 GiB Speicher ausgestattet ist.

Wie die obere Hälfte der Tabelle~\ref{table:paris_vs_scherl_zhao} zeigt, kann die vorgestellte Implementierung die von
Scherl et al.\ erreichten Werte, inklusive der Datenein- und Ausgabe, auf modernerer Hardware um ein Viertel
unterbieten. Es ist ebenfalls gelungen, die von Zhao et al.\ vorgestellten Zeitangaben auf bis zu ein Drittel der Zeit
zu reduzieren.

\begin{table}
    \centering
    \begin{tabular}{llccc}
        \hline
        & GPU & Volumengröße [Voxel] & Projektionszahl & Zeitbedarf [s]\\
        \hline
        Scherl et al. & GeForce 8800 GTX & 512 x 512 x 512 & 414 & 12\\
        Stephan & GeForce GTX 1080 & 512 x 512 x 512 & 360 & 9\\
        Stephan & GeForce GTX 1080 & 512 x 512 x 512 & 480 & 9\\
        \hline
        Zhao et al. & Quadro FX4600 & 512 x 512 x 512 & 360 & 7,7\\
        Stephan & GeForce GTX 1080 & 512 x 512 x 512 & 360 & 2\\
        Zhao et al. & Quadro FX4600 & 1024 x 1024 x 1024 & 720 & 101,9\\
        Stephan & GeForce GTX 1080 & 1024 x 1024 x 1024 & 720 & 34\\
        \hline
    \end{tabular}
    \caption{Vergleich mit denen von Scherl et al.\ und Zhao et al.\ vorgestellten Ansätzen}
    \label{table:paris_vs_scherl_zhao}
\end{table}

\chapter{Zusammenfassung und Ausblick}

Es wurde in dieser Arbeit eine mögliche Implementierung des \gls{fdk} auf dem Fundament der CUDA-Plattform vorgestellt.
Das Ziel, die Rückprojektion in sinnvoller Zeit zu berechnen, ist erreicht worden. Abhängig von der gewünschten Größe
des Volumens benötigt der Algorithmus auf moderner Hardware wenige Sekunden bis Minuten. Die Wartezeit zwischen zwei
Rückprojektionen konnte auf wenige Millisekunden reduziert werden. Insbesondere bei großen Volumen ist sie dadurch
vernachlässigbar gering, da sie, akkumuliert über die gesamte Laufzeit von einigen Minuten, lediglich wenige Sekunden
einnimmt. Bei der Rekonstruktion kleinerer Volumen fällt sie allerdings stärker ins Gewicht.

Es wurde ebenfalls erreicht, dass der Rückprojektions-\gls{kernel} die \gls{gpu} möglichst stark auslastet. Außerdem
gelang es, den Algorithmus so zu implementieren, dass er bei der Berechnung großer Volumen von mehreren \gls{gpu}s des
gleichen Typs beschleunigt wird. Die Rekonstruktion kleinerer Volumen profitiert allerdings nicht vom Einsatz mehrerer
Grafikkarten; eine einzelne \gls{gpu} war in allen Fällen schneller. Die Verwendung verschiedener \gls{gpu}s kann
außerdem dazu führen, dass der Algorithmus langsamer ausgeführt wird, als wenn man nur die leistungsstärkere Grafikkarte
verwendet hätte.

Die Verwendung von CUDA hat sich für die Parallelisierung des \gls{fdk} als vorteilhaft erwiesen. Durch die massiv
datenparallele Philosophie eignet sich CUDA gut für die Berechnung unabhängiger \gls{voxel}. Aufgrund der
Spezialisierung der \gls{gpu}s auf grafische Operationen sind sie gut für den Einsatz in bildgebenden Messverfahren wie
der Computertomographie geeignet. Eine Hürde, die beachtet werden muss, bleibt allerdings der begrenzte Speicher,
besonders im Hinblick auf die von neuen und zukünftigen Computertomographieanlagen erzeugten Datenmengen.

Wie die Analyse gezeigt hat, ist die vorgestellte Implementierung noch nicht an allen Stellen optimal. Insbesondere ist
es bei kleinen Volumen nicht gelungen, die in der Literatur präsentierten Zeiten für die Rekonstruktion zu erreichen,
was vermutlich auf zu starken Host-Overhead zurückzuführen ist. Dieser Overhead steigt durch den Einsatz mehrerer
\gls{gpu}s sogar noch weiter an, sodass der nächste Optimierungsschritt des Programms darin bestehen muss, die
Operationen auf der \gls{host}-Seite zu optimieren.

Die Skalierung des \gls{fdk} über mehrere \gls{gpu}s ist insgesamt verbesserungswürdig. Die Tatsache, dass selbst bei
der Rekonstruktion großer Volumen der Einsatz zwei verschiedener \gls{gpu}s länger dauert als der alleinige Einsatz der
leistungsstärkeren Karte, zeigt deutlich die Limitierungen der verwendeten statischen Lastverteilung. Denkbar ist hier
eine vor der Rückprojektion stattfindende algorithmische Bestimmung der idealen Last pro \gls{gpu}, beispielsweise in
Abhängigkeit von der Taktfrequenz und dem verfügbaren Speicher der jeweiligen Grafikkarte. Auch der Einsatz von
\textit{Machine-Learning}-Techniken kann in Betracht gezogen werden.

Die gefilterte Rückprojektion könnte außerdem durch einige Ansätze aus der Literatur weiter verbessert werden. Bei
großen Volumen wäre die Ausnutzung der Symmetrien, wie sie von Zhao et al.\ vorgeschlagen wird, eine Möglichkeit zur
weiteren Geschwindigkeitssteigerung. Der Ansatz von Scherl et al.\, mit einem zweidimensionalen Kernel durch das Volumen
zu iterieren, könnte die Zahl der für die Rückprojektion benötigten Ressourcen reduzieren und so auf dem \gls{device}
die zur Rückprojektion parallele Ausführung der Wichtung und Filterung erlauben. Dadurch wäre es möglich, die Zeit
zwischen zwei Rückprojektionen weiter zu reduzieren.

Die vorgestellte Implementierung des \gls{fdk} berechnet die Rückprojektion für derzeit gängige Projektionsdatensätze
in wenigen Minuten. Aufgrund der Aufnahmegeschwindigkeit einer Computertomographieanlage ist es denkbar, die
generierten Projektionen in Echtzeit zu verarbeiten. Die für die Berechnung benötigte Zeit könnte so durch die Aufnahme
verdeckt werden, sodass am Ende der Aufnahme bereits ein rekonstruiertes Volumen vorliegt. Dabei wäre es möglich bzw.\
für die Echtzeitrekonstruktion notwendig, vor der Wichtung und der Filterung der Projektionen zusätzliche
Vorverarbeitungsschritte in den Algorithmus zu integrieren, wie etwa eine Korrektur defekter Pixel oder eine
automatische Berücksichtigung des Detektor-Offsets.

Das im Rahmen dieser Arbeit am Helmholtz-Zentrum Dresden-Rossendorf entstandene Programm wurde unter eine freie Lizenz
gestellt. Der Quelltext ist im Internet unter der nachstehenden Adresse verfügbar:
\url{https://www.github.com/HZDR-FWDF/PARIS}

\appendix

\chapter{Grundlagen der Computertomographie}

\section{Herleitung des Fourier-Scheiben-Theorems}\label{app:fourier_scheibe}

Seien $F(u, v)$ die zweidimensionale
Fouriertransformation des Objekts $f(x, y)$:

\begin{equation}\label{eq:obj_2d_full}
    F(u, v) = \int\limits_{-\infty}^{\infty} \int\limits_{-\infty}^{\infty} f(x, y) \cdot e^{-2 \pi i \cdot(u x + v y)}\
        \mathrm{d} x\ \mathrm{d} y
\end{equation}

und $S_{\theta}(w)$ die eindimensionale Fouriertransformation der Projektion unter dem Winkel $\theta$ $P_{\theta}(t)$:

\begin{equation}\label{eq:proj_fft}
    S_{\theta}(w) = \int\limits_{-\infty}^{\infty} P_{\theta}(t) \cdot e^{-2 \pi i \cdot w t}\ \mathrm{d} t
\end{equation}

Im Folgenden sei der Fall $\theta = 0$ betrachtet. Setzt man $v = 0$, so wird die Formel~\ref{eq:obj_2d_full}
vereinfacht:

\begin{equation}
    F(u, 0) = \int\limits_{-\infty}^{\infty} \int\limits_{-\infty}^{\infty} f(x, y) \cdot e^{-2 \pi i \cdot u x}\
    \mathrm{d} x\ \mathrm{d} y
\end{equation}

Der Phasenfaktor ist nun nicht mehr von $y$ abhängig, wodurch das Integral in zwei Hälften geteilt werden kann:

\begin{equation}\label{eq:obj_2d_split}
    F(u, 0) = \int\limits_{-\infty}^{\infty} \left[\int\limits_{-\infty}^{\infty} f(x, y)\ \mathrm{d} y \right] \cdot
    e^{-2 \pi i \cdot u x}\ \mathrm{d} x
\end{equation}

Der Term in eckigen Klammern entspricht der Gleichung für eine Projektion entlang konstanter $x$-Linien:

\begin{equation}
    P_{\theta = 0}(x) = \int\limits_{-\infty}^{\infty} f(x, y)\ \mathrm{d} y
\end{equation}

Durch Einsetzen in Gleichung~\ref{eq:obj_2d_split} erhält man:

\begin{equation}
    F(u, 0) = \int\limits_{-\infty}^{\infty} P_{\theta = 0}(x) \cdot e^{-2 \pi i \cdot u x}\ \mathrm{d} x
\end{equation}

Die rechte Seite dieser Gleichung stellt die eindimensional fouriertransformierte Projektion $P_{\theta = 0}$ dar (vgl.
Gleichung~\ref{eq:proj_fft}), es ergibt sich also der folgende Zusammenhang zwischen der transformierten Projektion und
dem zweidimensional fouriertransformierten Objekt:

\begin{equation}
    F(u, 0) = S_{\theta = 0}(u)
\end{equation}

Dieses Ergebnis ist die einfachste Form des Fourier-Scheiben-Theorems. Darüber hinaus ist es unabhängig von der
konkreten Konstellation zwischen dem Objekt und dem Koordinatensystem der Projektion. Wird das
$(t, s)$-Koordinatensystem um den Winkel $\theta$ rotiert, so ist die Fouriertransformation der Projektion gleich der
zweidimensionalen Fouriertransformation des Objekts entlang einer radialen Linie, die um den Winkel $\theta$ rotiert
wird (siehe Abbildung~\ref{fig:fourier_scheibe}, vgl.~\cite{rosenkak}, S. 366).

\chapter{Implementierung des FDK-Algorithmus}

\section{Implementierung der Filterung}

\begin{code}
\begin{minted}[breaklines,breakafter=\,,fontsize=\small]{cuda}
__global__ void filter_creation_kernel(float* __restrict__ r,
    const std::int32_t* __restrict__ j, std::uint32_t size, float tau)
{
    auto x = blockIdx.x * blockDim.x + threadIdx.x;

    if(x < size)
    {
        if(j[x] == 0)
            r[x] = (1.f / 8.f) * (1.f / powf(tau, 2.f));
        else
        {
            if(j[x] % 2 == 0)
                r[x] = 0.f;
            else
                r[x] = -(1.f / (2.f * powf(j[x], 2.f)
                        * powf(M_PI, 2.f)
                        * powf(tau, 2.f)));
        }
    }
}
\end{minted}
\captionof{listing}{Filtergenerierung}
\label{app:filter_gen}
\end{code}

\begin{code}
\begin{minted}[escapeinside=||,fontsize=\small]{cuda}
__global__ void normalization_kernel(cufftReal* dst,
    std::size_t pitch, std::uint32_t width,
    std::uint32_t height, std::uint32_t filter_size)
{
    auto x = blockIdx.x * blockDim.x + threadIdx.x;
    auto y = blockIdx.y * blockDim.y + threadIdx.y;

    if((x < width) && (y < height))
    {
        auto row = |\textbf{\textcolor{keyword-green}{reinterpret\_cast}}|<cufftReal*>(
            |\textbf{\textcolor{keyword-green}{reinterpret\_cast}}|<char*>(dst) + y * pitch);

        row[x] /= filter_size;
    }
}
\end{minted}
\captionof{listing}{Projektionsnormalisierung}
\label{app:filter_norm}
\end{code}

\section{Implementierung der gefilterten Rückprojektion}

\begin{code}
\begin{minted}[breaklines,breakafter=\,,fontsize=\small]{cuda}
__device__ auto vol_centered_coordinate(unsigned int coord,
                                        std::uint32_t dim,
                                        float size)
    -> float
{
    auto size2 = size / 2.f;
    return -(dim * size2) + size2 + coord * size;
}
\end{minted}
\captionof{listing}{Koordinatensystemtransformation im Volumen}
\label{app:coord_vol}
\end{code}

\begin{code}
\begin{minted}[breaklines,breakafter=\,,fontsize=\small]{cuda}
__device__ auto proj_pixel_coordinate(float coord, std::uint32_t dim,
                                      float size, float offset)
    -> float
{
    auto size2 = size / 2.f;
    auto min = -(dim * size2) - offset;
    return (coord - min) / size - (1.f / 2.f);
}
\end{minted}
\captionof{listing}{Koordinatensystemtransformation auf dem Detektor}
\label{app:coord_det}
\end{code}

\begin{code}
\begin{minted}[breaklines,breakafter=\,,fontsize=\small]{c++}
struct backprojection_constants
{
    /* Volumenkonstanten */
    std::uint32_t vol_dim_x;    // Voxelzahl in x-Richtung
    std::uint32_t vol_dim_y;    // Voxelzahl in y-Richtung
    std::uint32_t vol_dim_z;    // Voxelzahl in z-Richtung

    float l_vx_x;               // Voxelgröße in x-Richtung
    float l_vx_y;               // Voxelgröße in y-Richtung
    float l_vx_z;               // Voxelgröße in z-Richtung

    /* Projektionskonstanten */
    std::uint32_t proj_dim_x;   // Pixelzahl in x-Richtung
    std::uint32_t proj_dim_y;   // Pixelzahl in y-Richtung

    float l_px_x;               // Pixelgröße in x-Richtung
    float l_px_y;               // Pixelgröße in y-Richtung

    /* Detektorkonstanten */
    float delta_s;              // horizontaler Offset
    float delta_t;              // vertikaler Offset

    float d_so;                 // Abstand Quelle - Objekt
    float d_sd;                 // Abstand Quelle - Detektor
};
\end{minted}
\caption{Struktur der Rückprojektions-Konstanten}
\label{app:fdk_consts}
\end{code}

\begin{code}
\begin{minted}[breaklines,breakafter=\,,escapeinside=||,fontsize=\small]{cuda}
__global__ void backprojection_kernel(
    float* vol, std::size_t vol_pitch, cudaTextureObject_t proj,
    float angle_sin, float angle_cos)
{
    auto k = blockIdx.x * blockDim.x + threadIdx.x;
    auto l = blockIdx.y * blockDim.y + threadIdx.y;
    auto m = blockIdx.z * blockDim.z + threadIdx.z;

    if((k < consts.vol_dim_x) && (l < consts.vol_dim_y) &&
       (m < consts.vol_dim_z)) {
        // berechne gegenwärtige Schicht und Zeile
        auto slice_pitch = vol_pitch * consts.vol_dim_y;
        auto slice = |\textbf{\textcolor{keyword-green}{reinterpret\_cast}}|<char*>(vol) + m * slice_pitch;
        auto row = |\textbf{\textcolor{keyword-green}{reinterpret\_cast}}|<float*>(slice + l * vol_pitch);

        // lade alten Wert aus dem globalen Speicher
        auto old_val = row[k];

        // Koordinatensystemursprung in Volumenmittelpunkt verschieben
        auto x_k = vol_centered_coordinate(k, consts.vol_dim_x,
                                           consts.l_vx_x);
        auto y_l = vol_centered_coordinate(l, consts.vol_dim_y,
                                           consts.l_vx_y);
        auto z_m = vol_centered_coordinate(m, consts.vol_dim_z,
                                           consts.l_vx_z);

        // Koordinaten rotieren
        auto s = x_k * angle_cos + y_l * angle_sin;
        auto t = -x_k * angle_sin + y_l * angle_cos;

        // projiziere rotierte Koordinaten auf Detektor
        auto factor = consts.d_sd / (s + consts.d_so);
        auto h = proj_pixel_coordinate(t * factor, consts.proj_dim_x,
             consts.l_px_x, consts.delta_s) + 0.5f;
        auto v = proj_pixel_coordinate(z_m * factor,
             consts.proj_dim_y, consts.l_px_y, consts.delta_t) + 0.5f;

        // lies Projektionswert an dieser Stelle
        auto det = tex2D<float>(proj, h, v);

        // Rückprojektion
        auto u = -(consts.d_so / (s + consts.d_so));
        row[k] = old_val + 0.5f * det * u * u;
    }
}
\end{minted}
\captionof{listing}{Vollständiger Rückprojektions-\gls{kernel}}
\label{app:impl_bp}
\end{code}



\end{document}
