\newglossary{nomencl}{ncs}{nco}{Nomenklatur}

\newglossarystyle{nomenclstyle}{
    % Umgebung
    \renewenvironment{theglossary}
        {\begin{longtable}{llc}}
        {\end{longtable}}
    % Tabellenkopf
    \renewcommand*{\glossaryheader}{\bfseries Name & \bfseries Beschreibung & \bfseries Symbol \\\hline\endhead}
    % kein Abstand zwischen Gruppen
    \renewcommand*{\glsgroupheading}[1]{}
    \renewcommand*{\glossaryentryfield}[5]{
    \glstarget{##1}{##2}% Name
            & ##3 % Beschreibung
            & ##4 % Symbol
            \\ % Zeilenende
    }
    % Was zwischen den einzelnen Glossar-Gruppen zu tun ist
    \renewcommand*{\glsgroupskip}{\relax}
}

\makeglossaries

\newglossaryentry{kernel}{name=Kernel,
                          description={Programm, das auf einem Beschleuniger, wie etwa einer GPU, ausgeführt wird},
                          plural=Kernel}
\newglossaryentry{cuda}{name={CUDA},
                        description={NVIDIA CUDA, proprietäre Plattform für die Programmierung von Grafikkarten}}
\newglossaryentry{directx}{name={DirectX},
                           description={Microsoft DirectX, proprietäre Plattform für u.a. Grafikprogrammierung}}
\newglossaryentry{pixel}{name=Pixel,
                         description={Punkt in einem zweidimensionalen Koordinatensystem, z.B.\ einem Bild oder einem
                                      Detektor},
                         plural=Pixel}
\newglossaryentry{voxel}{name=Voxel,
                         description={Punkt in einem dreidimensionalen Koordinatensystem, z.B.\ einem Volumen},
                         plural=Voxel}
\newglossaryentry{host}{name=Host,
                        description={Gerät, das einen Kernel auf dem Device ausführt. Üblicherweise ist dies das Gerät,
                                     auf dem auch das Betriebssystem läuft, etwa ein Rechner oder ein Knoten auf einem
                                     Superrechner},
                        plural=Hosts}
\newglossaryentry{device}{name=Device,
                          description={Beschleuniger, der einen Kernel ausführt. Im Zusammenhang mit CUDA ist dieser
                                       typischerweise eine GPU},
                          plural=Devices}
\newglossaryentry{stream}{name=Stream,
                          description={Warteschlange auf einem CUDA-Device. Operationen, wie z.B.\ Kernelaufrufe, werden
                                       innerhalb eines Streams sequentiell ausgeführt. Mehrere Streams können vom
                                       gleichen Device parallel abgearbeitet werden},
                          plural=Streams}
\newglossaryentry{race-condition}{name=Race Condition,
                                  description={Paralleler, nicht (ausschließlich) lesender Zugriff mehrerer Threads auf
                                               das gleiche Datum. Da die Reihenfolge der Schreib- und Lesezugriffe ohne
                                               weitere Synchronisierungsmechanismen nicht definiert ist, besteht die
                                               Möglichkeit der Datenkorruption},
                                  plural=race conditions}
\newglossaryentry{grid}{name=Grid,
                        description={Menge aller auf dem Device durch einen Kernel gestarteten Threads},
                        plural=Grids}

\newacronym{fdk}{FDK-Algorithmus}{Feldkamp-Davis-Kress-Algorithmus}
\newacronym{hzdr}{HZDR}{Helmholtz-Zentrum Dresden-Rossendorf}
\newacronym{paris}{PARIS}{\textit{Portable and Accelerated 3D Reconstruction tool for radiation based Imaging
                                  Systems}}
\newacronym{fpga}{FPGA}{\textit{Field Programmable Gate Array}}
\newacronym{opencl}{OpenCL}{\textit{Open Computing Language}}
\newacronym{fpu}{FPU}{\textit{floating point unit}}
\newacronym{gpgpu}{GPGPU}{\textit{General Purpose Computation on Graphics Processing Unit}}
\newacronym{opengl}{OpenGL}{\textit{Open Graphics Library}}
\newacronym{openmp}{OpenMP}{\textit{Open Multi-Processing}}
\newacronym{openacc}{OpenACC}{\textit{Open Accelerators}}
\newacronym{gpu}{GPU}{\textit{Graphics Processing Unit}}
\newacronym{cpu}{CPU}{\textit{Central Processing Unit}}
\newacronym{simd}{SIMD}{\textit{Single Instruction, Multiple Data}}
\newacronym{simt}{SIMT}{\textit{Single Instruction, Multiple Threads}}
\newacronym{sm}{SM}{\textit{Streaming Multiprocessor}}
\newacronym{gpc}{GPC}{\textit{Graphics Processing Cluster}}
\newacronym{cufft}{cuFFT}{NVIDIA \gls{cuda} \textit{Fast Fourier Transform}}
\newacronym{hpc}{HPC}{\textit{High Performance Computing}}

\newglossaryentry{proj}{name=Projektion,
                        description={Röntgenaufnahme eines Objekts},
                        symbol={$p$},
                        plural=Projektionen,
                        type=nomencl}
\newglossaryentry{fourproj}{name={fouriertransformierte Projektion},
                            description={in den Frequenzraum überführte Projektion},
                            symbol={$P$},
                            plural={fouriertransformierte Projektionen},
                            sort={Projektion, fouriertransformiert},
                            type=nomencl}
\newglossaryentry{filtproj}{name={gefilterte Projektion},
                            description={im Frequenzraum gewichtete Projektion},
                            symbol={$p_F$},
                            plural={gefilterte Projektionen},
                            sort={Projektion, gefiltert},
                            type=nomencl}
\newglossaryentry{obj}{name=Objekt,
                       description={von Röntgenstrahlen durchleuchtetes Objekt},
                       symbol={$f$},
                       plural=Objekte,
                       type=nomencl}
\newglossaryentry{fourobj}{name={fouriertransformiertes Objekt},
                           description={in den Frequenzraum überführtes Objekt},
                           symbol={$F$},
                           plural={fouriertransformierte Objekte},
                           sort={Objekt, fouriertransformiert},
                           type=nomencl}
\newglossaryentry{vol}{name=Volumen,
                       description={aus Projektionen rekonstruiertes Objekt},
                       symbol={$v$},
                       plural=Volumina,
                       type=nomencl}
\newglossaryentry{dsrc}{name=Quelle-Objekt-Abstand,
                        description={Abstand zwischen Quelle und Objektmittelpunkt},
                        symbol={$d_{src}$},
                        sort={Abstand, Quelle-Objekt},
                        type=nomencl}
\newglossaryentry{ddet}{name=Objekt-Detektor-Abstand,
                        description={Abstand zwischen Objektmittelpunkt und Detektor},
                        symbol={$d_{det}$},
                        sort={Abstand, Objekt-Detektor},
                        type=nomencl}
\newglossaryentry{dsd}{name=Quelle-Detektor-Abstand,
                       description={Abstand zwischen Quelle und Detektor},
                       symbol={$d_{sd}$},
                       sort={Abstand, Quelle-Detektor},
                       type=nomencl}
\newglossaryentry{nh}{name=Projektionszeilenlänge,
                      description={Pixelzahl pro Projektionszeile},
                      symbol={$N_h$},
                      sort={Länge, Projektionszeile},
                      type=nomencl}
\newglossaryentry{nv}{name=Projektionsspaltenlänge,
                      description={Pixelzahl pro Projektionsspalte},
                      symbol={$N_v$},
                      sort={Länge, Projektionsspalte},
                      type=nomencl}
\newglossaryentry{dh}{name={Pixelbreite},
                      description={physische Breite eines Detektorpixels},
                      symbol={$d_h$},
                      sort={Pixelabstand, horizontal},
                      type=nomencl}
\newglossaryentry{dv}{name={Pixelhöhe},
                      description={physische Höhe eines Detektorpixels},
                      symbol={$d_v$},
                      sort={Pixelabstand, vertikal},
                      type=nomencl}
\newglossaryentry{nx}{name=Volumenzeilenlänge,
                      description={Voxelzahl pro Volumenzeile},
                      symbol={$N_x$},
                      sort={Länge, Volumenzeile},
                      type=nomencl}
\newglossaryentry{ny}{name=Volumenspaltenlänge,
                      description={Voxelzahl pro Volumenspalte},
                      symbol={$N_y$},
                      sort={Länge, Volumenspalte},
                      type=nomencl}
\newglossaryentry{nz}{name=Volumenschichten,
                      description={Anzahl der Schichten im Volumen},
                      symbol={$N_z$},
                      sort={Länge, Volumenschichten},
                      type=nomencl}
\newglossaryentry{dx}{name={Voxelbreite},
                      description={physische Breite eines Voxels},
                      symbol={$d_x$},
                      sort={Voxelabstand, horizontal},
                      type=nomencl}
\newglossaryentry{dy}{name={Voxelhöhe},
                      description={physische Höhe eines Voxels},
                      symbol={$d_y$},
                      sort={Voxelabstand, vertikal},
                      type=nomencl}
\newglossaryentry{dz}{name={Voxeltiefe},
                      description={physische Tiefe eines Voxels},
                      symbol={$d_z$},
                      sort={Voxelabstand, Tiefe},
                      type=nomencl}
\newglossaryentry{deltah}{name={horizontaler Detektoroffset},
                          description={horizontale Verschiebung des Detektors},
                          symbol={$\Delta h$},
                          sort={Detektoroffset, horizontal},
                          type=nomencl}
\newglossaryentry{deltav}{name={vertikaler Detektoroffset},
                          description={vertikale Verschiebung des Detektors},
                          symbol={$\Delta v$},
                          sort={Detektoroffset, vertikal},
                          type=nomencl}



\glsadd{filtproj}
\glsadd{vol}
\glsadd{dsd}
\glsadd{nh}
\glsadd{dh}
\glsadd{dv}
\glsadd{nx}
\glsadd{ny}
\glsadd{dx}
\glsadd{deltah}
\glsadd{deltav}
