\chapter{Grundlagen}

\section{Die Computertomographie}

\subsection{Die Geschichte und prinzipielle Funktionsweise der Computertomographie}

Die Geschichte der Computertomographie beginnt mit dem Röntgenverfahren, welches 1895 vom deutschen Physiker Wilhelm
Conrad Röntgen entdeckt wurde~\cite{roentgen}. Mittels einer Strahlungsquelle wird ein Objekt durchleuchtet und auf
einem Film bzw.\ einem Detektor abgebildet; der dreidimensionale Körper wird also auf eine zweidimensionale Fläche
projiziert. Diesen Schritt bezeichnet man als \textit{Vorwärtsprojektion}.

Führt man die Vorwärtsprojektion genügend oft in aufeinanderfolgenden Winkelschritten aus, bis man (idealerweise) einen
Vollkreis abgefahren hat, so lässt sich aus den dabei entstandenen \textit{Projektionen} der ursprünglich durchleuchtete
Körper, den wir in der Folge als \textit{Volumen} bezeichnen, rekonstruieren. Für jeden Punkt im Volumen
(\textit{Voxel}) kann anhand der Informationen aus den Projektionen der Absorptionsgrad berechnet und dadurch die
innere Struktur des Volumens bestimmt werden. Dieser Zusammenhang wurde in den 60er Jahren des 20. Jahrhunderts durch
den südafrikanisch-amerikanischen Physiker Allan McLeod Cormack festgestellt, der ebenfalls die dazu notwendigen
mathematischen Grundlagen entwickelte~\cite{cormack63}~\cite{cormack64}; ihm war allerdings unbekannt~\cite{cormack79},
dass diese schon 1917 vom österreichischen Mathematiker Johann Radon gefunden wurden~\cite{radon}. Mathematisch ist der
Vorgang der \textit{Rückprojektion} eine Anwendung der nach Radon benannten \textit{Radon-Transformation}.

Ein Problem der Vorwärtsprojektion ist der Informationsverlust, der durch die mangelnde Tiefe des Detektors entsteht;
die Tiefeninformationen werden auf die zweidimensionale Fläche {\glqq}verschmiert{\grqq}. Bei der Rückprojektion lässt
sich dieser Verlust durch die Wahl eines geeigneten Bildfilters wiederum kaschieren, weshalb man auch von der
\textit{gefilterten Rückprojektion} spricht.

Da die gefilterte Rückprojektion für jedes Voxel einzeln berechnet werden muss, ist sie für einen Menschen nicht in
sinnvoller Zeit lösbar. Aus diesem Grund ist man für die Lösung des Gesamtproblems also auf einen Computer angewiesen,
woraus sich der Name des Verfahrens ableitet: \textit{Computertomographie}. Die ersten bis zur Marktreife entwickelten
Computertomographen wurden gegen Ende der 60er Jahre des 20. Jahrhunderts vom englischen Elektroingenieur Godfrey
Hounsfield gebaut. Dieser entwickelte die für die Rückprojektion nötigen Algorithmen ebenfalls selbst, da ihm die
Vorarbeiten von Cormack und Radon nicht bekannt waren~\cite{kalender}. Für ihre voneinander unabhängigen Arbeiten
erhielten Godfrey und Cormack 1979 den Nobelpreis für Physiologie oder Medizin, was die Bedeutung der
Computertomographie insbesondere für die Medizin unterstreicht.

\subsection{Der Feldkamp-Davis-Kress-Algorithmus und seine Parallelisierung}

Der 1984 entwickelte \gls{fdk}~\cite{fdk} ist eine spezielle Ausprägung der gefilterten Rückprojektion. Ausgangspunkt
der Strahlung ist eine Quelle, die das Volumen mit einem \textit{kegelförmigen} Strahl durchleuchtet und damit auf einem
Detektor abbildet. Der Vorteil des \gls{fdk} liegt darin, dass die gefilterte Rückprojektion für jedes Voxel individuell
berechnet werden kann, das heißt ohne Abhängigkeiten zu anderen Voxeln. Dieser Umstand ermöglicht einen Grad an
Parallelität, für den sich im Englischen der Begriff \textit{embarassingly parallel} eingebürgert hat und macht den
\gls{fdk} zu einem idealen Ziel für diverse Parallelisierungsansätze.

\begin{itemize}
    \item Xu et al.~\cite{xumuell}: vermutlich OpenGL
    \item Li et al.~\cite{lipapa}: FPGA
    \item Knaup et al.~\cite{knaupsteck}: PowerPC
    \item Scherl et al.~\cite{scherlkeck}: \gls{cuda}
    \item Balász et al.~\cite{balgab}: OpenCL
    \item Hofmann et al.~\cite{hoftrei}: Intel{\textregistered} Xeon Phi{\texttrademark}: Knights Corner
\end{itemize}

Welche Vor- und Nachteile haben diese Ansätze? -> Warum brauchen wir etwas neues / warum können wir sie nicht anwenden?

\section{Die NVIDIA{\textregistered}-CUDA{\textregistered}-Plattform}

\subsection{Grafikkarten und Wissenschaft}

\subsection{Das CUDA{\textregistered}-Programmiermodell}

\subsection{Alternativen zu CUDA{\textregistered}}
