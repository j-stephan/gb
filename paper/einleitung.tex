\chapter{Einleitung}

\section{Die Geschichte und Relevanz der Computertomographie}

Die Geschichte der Computertomographie beginnt mit dem vom deutschen Physiker Wilhelm Conrad Röntgen entdeckten
und später nach ihm benannten Verfahren der {\glqq}X-Strahlen{\grqq}~\cite{roentgen}. Es war nun möglich, die innere
Beschaffenheit eines Objekts auf nichtinvasive Art und Weise -- also ohne es zu beschädigen -- festzustellen. Die
Bedeutung dieses Verfahrens insbesondere für die Anwendung in der Medizin war bereits Röntgens Zeitgenossen klar.
So druckte die Wiener Zeitung {\glqq}Die Presse{\grqq} am 05.\ Januar 1896 auf ihrer Titelseite unter der Überschrift
{\glqq}Eine sensationelle Entdeckung{\grqq}: {\glqq}[Man hat es] mit einem in seiner Art epochemachenden Ergebnisse der
exacten Forschung zu thun, das sowol[sic] auf physikalischem wie auf medicinischem Gebiete ganz merkwürdige Consequenzen
bringen dürfte.{\grqq} Für seine Entdeckung wurde Röntgen in der Folge unter anderem mit dem ersten Nobelpreis für
Physik ausgezeichnet. Bis heute ist das Röntgenverfahren ein wichtiger Bestandteil der medizinischen Diagnostik und der
Werkstoffprüfung.

\section{Der Einsatz der Computertomographie am Helmholtz-Zentrum Dresden-Rossendorf}

\begin{itemize}
    \item Ausgangssituation am HZDR:\@ Uraltes FDK-Programm braucht mehrere Tage für eine Rekonstruktion (schlecht)
    \item Gesamtziel: Dieses Programm soll durch ein schnelleres und möglichst gut optimiertes abgelöst werden
\end{itemize}

\section{Aufgabenstellung}

Der \gls{fdk} ist ein weit verbreiteter Ansatz zur Rekonstruktion von kegelförmiger Computer-Tomographie. In diesem
Beleg soll untersucht werden:

\begin{itemize}
    \item Zusammenfassung des Forschungsstandes hinsichtlich der Parallelisierung / der Verwendung von \gls{cuda}
    \item Gegenüberstellung verschiedener Optimierungsziele (Time-to-solution, Occupancy)
    \item Variantenvergleich verschiedener Implementierungsstrategien
    \item Implementierung und Analyse einer dieser Strategien
\end{itemize}
